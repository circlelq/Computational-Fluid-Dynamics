\documentclass[12pt]{article}
\input{/Users/circle/Documents/博一下/homework/setting.tex}
\setcounter{secnumdepth}{2}
\usepackage{bm}
\usepackage{autobreak}
\usepackage{amsmath}
\setlength{\parindent}{2em}
\graphicspath{{../}}

%pdf文件设置
\hypersetup{
	pdfauthor={袁磊祺},
	pdftitle={计算流体力学作业1}
}

\title{
		\vspace{-1in} 	
		\usefont{OT1}{bch}{b}{n}
		\normalfont \normalsize \textsc{\LARGE Peking University}\\[1cm] % Name of your university/college \\ [25pt]
		\horrule{0.5pt} \\[0.5cm]
		\huge \bfseries{计算流体力学作业1} \\
		\horrule{2pt} \\[0.5cm]
}
\author{
		\normalfont 								\normalsize
		College of Engineering \quad 2001111690  \quad 袁磊祺\\	\normalsize
        \today
}
\date{}

\begin{document}

%%%%%%%%%%%%%%%%%%%%%%%%%%%%%%%%%%%%%%%%%%%%%%
\captionsetup[figure]{name={图},labelsep=period}
\captionsetup[table]{name={表},labelsep=period}
\renewcommand\contentsname{目录}
\renewcommand\listfigurename{插图目录}
\renewcommand\listtablename{表格目录}
\renewcommand\refname{参考文献}
\renewcommand\indexname{索引}
\renewcommand\figurename{图}
\renewcommand\tablename{表}
\renewcommand\abstractname{摘\quad 要}
\renewcommand\partname{部分}
\renewcommand\appendixname{附录}
\def\equationautorefname{式}%
\def\footnoteautorefname{脚注}%
\def\itemautorefname{项}%
\def\figureautorefname{图}%
\def\tableautorefname{表}%
\def\partautorefname{篇}%
\def\appendixautorefname{附录}%
\def\chapterautorefname{章}%
\def\sectionautorefname{节}%
\def\subsectionautorefname{小小节}%
\def\subsubsectionautorefname{subsubsection}%
\def\paragraphautorefname{段落}%
\def\subparagraphautorefname{子段落}%
\def\FancyVerbLineautorefname{行}%
\def\theoremautorefname{定理}%
\crefname{figure}{图}{图}
\crefname{equation}{式}{式}
\crefname{table}{表}{表}
%%%%%%%%%%%%%%%%%%%%%%%%%%%%%%%%%%%%%%%%%%%

\maketitle

\section{简介}

流体力学在我们生活中无处不在,小到毛细血管里的流动,大到气象运动、火星上点涡运动形成的绕圈现象,以及银河的运动演化. 我们时刻处于空气这种流体中,流体力学和我们的生活息息相关,例如精彩的香蕉球,就靠足球和空气的相互作用来使得足球绕弯,除了足球,网球也能看到这种现象,网球球星纳达尔就偶尔会打出绕弯的球,本来是出界的球可以绕进场内. 

\section{连续介质假设}

研究流体时一个很重要的假设就是连续介质假设,认为流体质点连续地充满了流体所在的整个空间. 流体质点所具有的宏观物理量(如质量、速度、压力、温度等)满足一起应该遵循的物理定律及物理性质,例如牛顿定律、质量、能量守恒定律、热力学定律,以及扩散、粘性、热传导等疏运性质. 但流体但某些物理常数和关系还必须由实验确定. 


\section{流体的性质}


流体在静止时不能承受切向应力,不管多小的切向应力,这是流体区别于固体的一个重要性质. 但是在一些情况下固体和流体的划分并不明显,例如胶状物和油漆这类触变物质防止一段事件后,他们的性质看起来想弹性固体,但是在摇动和刷漆时却失去弹性,发生很大但变形,其行为完全像流体. 沥青在正常条件下像固体,用锤子锤它会发生破裂,但是放在地面上在重力的作用下经过相当长的时间之后,会逐渐向四周铺开,它的行为又像流体. 

所以说流体和固体有时候并不好区分,但是对于我们一般研究的流体来说,例如水和空气,都非常好地符合不能承受切向应力的性质. 

\section{黏性}

虽然流体在静止时不能承受切应力,但是在运动时,对相邻两层流体间的相对运动是有抵抗的,黏性在流体中影响巨大. 当流体流过一个物体时,不管黏性系数有多小,物体表面都有一层边界层,在湍流边界层靠近壁面的地方,有一层黏性底层,即使它们的厚度很小,对流体的流动性质也是至关重要的. 例如面涡的脱落,流体分离的形成. 当黏性为0时,流体是理想流体,通常有稳定解,而当黏性趋于0时,流体却形成了湍流. 




\section{感想}



流体力学在工程中的各个领域都有应用,尤其是国家大型军事工程,例如航空中的飞机、高超声速战斗机,飞机外形的设计直接影响到飞机的飞行性能. 普通的客机机翼翼弦比较大,因为其速度较慢,需要运载的物体较重,需要较大的升力,而战斗机的翼弦比较小,多为三角翼,此时气体的流动特性并不相同. 客机要防止流体分离、失速,而战斗机的三角翼卷起的涡必定使得翼面上会有分离. 另外,在高超声速下,还会出现激波. 在进气道内还会出现多次激波反射,情况非常复杂. 


中国研发了高性能的导弹,例如东风-41弹道导弹. 在导弹发射过程中,也有很多流体相关的问题,首先导弹要快,尤其是洲际导弹要达到超高音速,这样敌军才无法拦截导弹. 另外还可以设计跳弹,即在导弹下坠的过程中,由于空气密度的变化发生打水飘的效应,使得敌军更加无法预测导弹的轨迹. 

除此之外,还有核潜艇,需要考虑固壁与水的相互作用,怎样才能更安静地在水里潜行而不被敌军发现. 以及星舰,在星舰下落过程中通过翼来调整姿态,使星舰水平下落,增大阻力. 

中国最近几天才实现全面脱贫,还是发展中国家,多少还是会受到别的国家的军事压力,很大程度上还需要把国家研究重点放在航空航天、导弹等国防事业上,所以中国的建设特别需要流体力学的发展,流体力学有着举足轻重作用. 






% \nocite{*}

\bibliographystyle{plain}

\phantomsection

\addcontentsline{toc}{section}{参考文献} %向目录中添加条目,以章的名义
\bibliography{homework}

\end{document}
