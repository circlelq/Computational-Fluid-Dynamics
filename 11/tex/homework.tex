\documentclass[12pt]{article}
\input{/Users/circle/Documents/博一下/homework/setting.tex}
\setcounter{secnumdepth}{2}
\usepackage{autobreak}
\usepackage{amsmath}
\setlength{\parindent}{2em}
\graphicspath{{../}}
\ziju{0.1pt}

%pdf文件设置
\hypersetup{
	pdfauthor={袁磊祺},
	pdftitle={计算流体力学作业11}
}

\title{
		\vspace{-1in} 	
		\usefont{OT1}{bch}{b}{n}
		\normalfont \normalsize \textsc{\LARGE Peking University}\\[0.2cm] % Name of your university/college \\ [25pt]
		\horrule{0.5pt} \\[0.2cm]
		\huge \bfseries{计算流体力学作业11} \\[-0.2cm]
		\horrule{2pt} \\[0.2cm]
}
\author{
		\normalfont 								\normalsize
		College of Engineering \quad 2001111690  \quad 袁磊祺\\	\normalsize
        \today
}
\date{}

\begin{document}

\input{/Users/circle/Documents/博一下/homework/setc.tex}

\maketitle

\section{A}

讲义 (CFDLect06-com0_cn.pdf) 中给出了完全气体Euler方程组和等温方程组的Steger-Wearming 通量向量分裂格式, 是否能给出等熵方程组(见讲义 (CFDLect06-com03_cn.pdf) 的第12页页底)的Steger-Wearming 通量向量分裂格式?

考虑气体动力学方程组
\begin{equation}
	\begin{pmatrix}
		\rho \\
		\rho u
	\end{pmatrix}_t
	+\begin{pmatrix}
		\rho u \\
		\rho u^2+\rho^\gamma/\gamma
	\end{pmatrix}_x = 0,\quad p(\rho) = \rho^\gamma/\gamma.
	\label{eq:11}
\end{equation}
对光滑解,\cref{eq:11}等价于
\begin{equation}
	\bm{U}_t + \nabla \bm{F}(\bm{U})\bm{U}_x=0,
\end{equation}
其中
\begin{equation}
	\bm{A} = \nabla \bm{F}(\bm{U}) = \begin{pmatrix}
		0       & 1  \\
		-u^2+p' & 2u
	\end{pmatrix},\quad \bm{U}=\begin{pmatrix}
		\rho \\
		\rho u
	\end{pmatrix}.
\end{equation}
\cref{eq:11}的特征值矩阵是
\begin{equation}
	\bm{\Lambda} =
	\begin{pmatrix}
		u - a & 0   \\
		0     & u+a
	\end{pmatrix}.
\end{equation}
其中
\begin{equation}
	a=\sqrt{p'},
\end{equation}
相应的特征向量矩阵是
\begin{equation}
	\bm{R} =
	\begin{pmatrix}
		1   & 1   \\
		u-a & u+a
	\end{pmatrix}.
\end{equation}
\begin{equation}
	\bm{L}= \begin{pmatrix}
		1 & u-a \\
		1 & u+a
	\end{pmatrix}.
\end{equation}
\begin{equation}
	\bm{L^}{-1} = \frac{1}{2a} \begin{pmatrix}
		u+a & -u+a \\
		-1  & 1
	\end{pmatrix}.
\end{equation}
所以
\begin{equation}
	\bm{A} = \bm{L}^{-1}\bm{\Lambda} \bm{L}.
\end{equation}
Steger-Wearming 通量向量分裂
\begin{equation}
	\bm{A} = \bm{A}^+ + \bm{A}^-,
\end{equation}
其中
\begin{equation}
	\bm{A}^+ = \frac{1}{2} \bm{L^}{-1}(\bm{\Lambda}+\abs{\bm{\Lambda}}) \bm{L },\quad \bm{A}^- = \frac{1}{2} \bm{L^}{-1}(\bm{\Lambda}-\abs{\bm{\Lambda}}) \bm{L^}.
\end{equation}



\section{B}

尝试给出等熵方程组(同上) 的Roe矩阵及Roe格式/方法.

Roe格式
\begin{equation}
	\hat{\bm{F}}\left(\bm{U}_{j}, \bm{U}_{j+1}\right)=\frac{\bm{F}\left(\bm{U}_{j}\right)+\bm{F}\left(\bm{U}_{j+1}\right)}{2}-\frac{1}{2}\left|\hat{\bm{A}}_{j+1 / 2}\right|\left(\bm{U}_{j+1}-\bm{U}_{j}\right).
\end{equation}
其中
\begin{equation}
	\hat{\bm{A}}(\bm{U}_l,\bm{U}_r) = \bm{A}(\bm{\bar{U}}),
\end{equation}
\begin{equation}
	\bm{F}=
	\begin{pmatrix}
		\rho u \\
		\rho u^2+\rho^\gamma/\gamma
	\end{pmatrix},
\end{equation}
其中
\begin{equation}
	\bm{\bar{U}}=
	\left\{
	\begin{aligned}
		\bar{\rho} & =\frac{\sqrt{\rho_{r}} \rho_{\ell}+\sqrt{\rho_{\ell}} \rho_{r}}{\sqrt{\rho_{\ell}}+\sqrt{\rho_{r}}}=\sqrt{\rho_{\ell} \rho_{r}}, \\
		\bar{u}    & =\frac{\sqrt{\rho_{\ell}} u_{\ell}+\sqrt{\rho_{r}} u_{r}}{\sqrt{\rho_{\ell}}+\sqrt{\rho_{r}}}.
	\end{aligned}\right.
\end{equation}
而
\begin{equation}
	\abs{\hat{\bm{A}}} = \bm{L^}{-1}\abs{\bm{\Lambda}} \bm{L}
\end{equation}
这里的$\bm{L},\ \bm{\Lambda}$为$\hat{\bm{A}}$的左特征向量矩阵和特征值矩阵。







\section{C}

尝试给出等温方程组(见讲义 (CFDLect06-com03_cn.pdf) 的第84页)和等熵方程组(同上) 的动理学通量向量分裂格式.

等温方程
\begin{equation}
	\begin{pmatrix}
		\rho \\
		\rho u
	\end{pmatrix}_t
	+\begin{pmatrix}
		\rho u \\
		\rho u^2+\rho a^2
	\end{pmatrix}_x = 0,\quad p(\rho) = \rho a^2.
	\label{eq:31}
\end{equation}

Kinetic flux-vector splitting: KFVS:
\begin{equation}
	F^{\pm}=\int v^{\pm} \Psi g d v d \xi=\rho\left(\begin{array}{c}
			<v^{1}>_{\pm} \\
			<v^{2}>_{\pm} \\
			\frac{1}{2}<v^{3}>_{\pm}+\frac{K}{2 \lambda}<v^{1}>_{\pm}
		\end{array}\right)
	\label{eq:32}
\end{equation}


其中 $v^{\pm}=\frac{1}{2}(v \pm|v|)$, $<v^{0}>_{\pm}=\frac{1}{2} \operatorname{erfc}(\mp \sqrt{\lambda} u),\ <v^{1}>_{\pm}=u<v^{0}>_{\pm} \pm \frac{\me^{-\lambda u^{2}}}{2 \sqrt{\pi \lambda}}$, $<v^{m+2}>_{\pm}=u<v^{m+1}>_{\pm}+\frac{m+1}{2 \lambda}<v^{m}>_{\pm},$ 其中$ \operatorname{erfc}(x)=\frac{2}{\sqrt{\pi}} \int_{x}^{\infty} \me^{-t^{2}} \dif t$, 其中$\lambda$是温度的函数,
\begin{equation}
	\lambda=\frac{(K+d) \rho}{4\left(E-\frac{1}{2} \rho \bm{u}^{2}\right)}=\frac{\rho}{2 p}.
\end{equation}
对于等温方程组和等熵方程组有
\begin{equation}
	\begin{pmatrix}
		\rho \\
		\rho u
	\end{pmatrix}_t
	+\begin{pmatrix}
		\rho u \\
		\rho u^2+p
	\end{pmatrix}_x = 0.
\end{equation}

\cref{eq:32}变为
\begin{equation}
	F^{\pm}=\int v^{\pm} \Psi g d v d \xi=\rho\left(\begin{array}{c}
			<v^{1}>_{\pm} \\
			<v^{2}>_{\pm}
		\end{array}\right),
\end{equation}
\begin{equation}
	<v^{0}>_{\pm}=\frac{1}{2} \operatorname{erfc}(\mp \sqrt{\lambda} u),
\end{equation}
\begin{equation}
	<v^{1}>_{\pm}=u<v^{0}>_{\pm} \pm \frac{\me^{-\lambda u^{2}}}{2 \sqrt{\pi \lambda}},
\end{equation}
\begin{equation}
	<v^{2}>_{\pm}=u<v^{1}>_{\pm}+\frac{1}{2 \lambda}<v^{0}>_{\pm}.
\end{equation}

对于等温气体
\begin{equation}
	\lambda = \frac{\rho}{2p} = \frac{\rho}{2\rho a^2} = \frac{1}{2a^2}.
\end{equation}

对于等熵气体
\begin{equation}
	\lambda = \frac{\rho}{2p} = \frac{\rho}{2\rho^\gamma/\gamma} = \frac{\gamma \rho^{1-\gamma}}{2}.
\end{equation}



\section{D}

尝试给出等熵方程组(同上) 的HLL解法器和格式.

类似Godunov格式,计算初始函数的单元平均值:
\begin{equation}
	\bar{\bm{u}}_{j}^{0}=\frac{1}{h} \int_{x_{j-\frac{1}{2}}}^{x_{j+\frac{1}{2}}} \bm{u}(x, 0) \dif x.
\end{equation}
将方程离散为
\begin{equation}
	\left.\frac{\dif \bar{\bm{u}}}{\dif t}\right|_{j}+\frac{1}{h}\left(\bm{f}_{j+\frac{1}{2}}-\bm{f}_{j-\frac{1}{2}}\right)=0,
\end{equation}
近似 Godunov 方法的界面通量
\begin{equation}
	\bm{f}_{i+\frac{1}{2}}^{\text{hll}}=\left\{\begin{array}{ll}
		\bm{f}_{L},                                                                                & 0 \leq S_{L},            \\
		\frac{S_{R} \bm{f}_{L}-S_{L} \bm{f}_{R}+S_{L} S_{R}\left(u_{R}-u_{L}\right)}{S_{R}-S_{L}}, & S_{L} \leq 0 \leq S_{R}, \\
		\bm{f}_{R},                                                                                & 0 \leq S_{R}.
	\end{array}\right.
	\label{eq:41}
\end{equation}
其中
\begin{equation}
	\bm{f}_R = \bm{f}(\bar{\bm{u}}_r),\quad \bm{f}_L = \bm{f}(\bar{\bm{u}}_l)
\end{equation}
对HLL方法需要估计 $S_{L}, S_{R} .$ 基本上有两个方法估 计 $S_{L}$ 和 $S_{R} .$ 最受欢迎的是直接估计速度. 较最近的方法取决于星型区内的 压カ、速度的估计, 然后利用精确波关系得到 $S_{L},\ S_{R} .$

方法一: 直接波速估计,例如
\begin{equation}
	S_{L}=u_{L}-a_{L}, \quad S_{R}=u_{R}+a_{R},
	\label{eq:42}
\end{equation}
和
\begin{equation}
	\begin{array}{l}
		S_{L}=\min \left\{u_{L}-a_{L}, \quad u_{R}-a_{R}\right\}, \\
		S_{R}=\max \left\{u_{L}+a_{L}, \quad u_{R}+a_{R}\right\}.
	\end{array}
\end{equation}

假设我们采用\cref{eq:42},则
\begin{equation}
	a_L = \sqrt{\bar{\rho}_L^{\gamma-1}},\quad a_R = \sqrt{\bar{\rho}_R^{\gamma-1}},
\end{equation}
\begin{equation}
	S_{L} = \bar{u}_L - a_L = \bar{u}_L - \sqrt{\bar{\rho}_L^{\gamma-1}},\quad S_{R} = \bar{u}_R + a_R = \bar{u}_R + \sqrt{\bar{\rho}_R^{\gamma-1}},
\end{equation}
将\cref{eq:42}代入\cref{eq:41}即可。下标$L = i,\ R = i+1$.




% \nocite{*}

% \input{bib.tex}

\end{document}

