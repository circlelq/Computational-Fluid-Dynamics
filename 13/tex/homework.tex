\documentclass[12pt]{article}
\input{/Users/circle/Documents/博一下/homework/setting.tex}
\setcounter{secnumdepth}{2}
\usepackage{autobreak}
\usepackage{amsmath}
\setlength{\parindent}{2em}
\graphicspath{{../}}
\ziju{0.1pt}

%pdf文件设置
\hypersetup{
	pdfauthor={袁磊祺},
	pdftitle={计算流体力学作业13}
}

\title{
		\vspace{-1in} 	
		\usefont{OT1}{bch}{b}{n}
		\normalfont \normalsize \textsc{\LARGE Peking University}\\[0.2cm] % Name of your university/college \\ [25pt]
		\horrule{0.5pt} \\[0.2cm]
		\huge \bfseries{计算流体力学作业13} \\[-0.2cm]
		\horrule{2pt} \\[0.2cm]
}
\author{
		\normalfont 								\normalsize
		College of Engineering \quad 2001111690  \quad 袁磊祺\\	\normalsize
        \today
}
\date{}

\begin{document}

%%%%%%%%%%%%%%%%%%%%%%%%%%%%%%%%%%%%%%%%%%%%%%
\captionsetup[figure]{name={图},labelsep=period}
\captionsetup[table]{name={表},labelsep=period}
\renewcommand\contentsname{目录}
\renewcommand\listfigurename{插图目录}
\renewcommand\listtablename{表格目录}
\renewcommand\refname{参考文献}
\renewcommand\indexname{索引}
\renewcommand\figurename{图}
\renewcommand\tablename{表}
\renewcommand\abstractname{摘\quad 要}
\renewcommand\partname{部分}
\renewcommand\appendixname{附录}
\def\equationautorefname{式}%
\def\footnoteautorefname{脚注}%
\def\itemautorefname{项}%
\def\figureautorefname{图}%
\def\tableautorefname{表}%
\def\partautorefname{篇}%
\def\appendixautorefname{附录}%
\def\chapterautorefname{章}%
\def\sectionautorefname{节}%
\def\subsectionautorefname{小小节}%
\def\subsubsectionautorefname{subsubsection}%
\def\paragraphautorefname{段落}%
\def\subparagraphautorefname{子段落}%
\def\FancyVerbLineautorefname{行}%
\def\theoremautorefname{定理}%
\crefname{figure}{图}{图}
\crefname{equation}{式}{式}
\crefname{table}{表}{表}
%%%%%%%%%%%%%%%%%%%%%%%%%%%%%%%%%%%%%%%%%%%

\maketitle

\section{A}

分别推导出二维可压缩Navier-Stokes(NS)方程(守恒形式) 和二维
不可压缩NS方程(原始变量形式)在可逆变换 $x=x(\xi, \eta), y=y(\xi, \eta)$ 下 的形式(散度形式和非散度形式)。如果考虑极坐标变换 $(x, y) \rightarrow (r, \theta)$, 则给出在 $(r, \theta)$ 坐标下方程的形式。举例说明,相应的边界条件变换后的形式。



假设坐标 $(\xi, \eta)$ 是与时间无关的。对于可逆变换有雅可比行列式
\begin{equation}
	J=
	\begin{vmatrix}
		\xi_{x}  & \xi_{y}  \\
		\eta_{x} & \eta_{y}
	\end{vmatrix}\not=0.
\end{equation}
根据矩阵关系
\begin{equation}
	\left(\begin{array}{ll}
			\xi_{x}  & \xi_{y}  \\
			\eta_{x} & \eta_{y}
		\end{array}\right)=\left(\begin{array}{ll}
		x_{\xi} & x_{\eta} \\
		y_{\xi} & y_{\eta}
	\end{array}\right)^{-1}
\end{equation}
可得
\begin{equation}
	\begin{aligned}
		x_{\xi}= & \frac{\eta_{y}}{J},  & x_{\eta}=    -\frac{\xi_{y}}{J}, \\
		y_{\xi}= & -\frac{\eta_{x}}{J}, & y_{\eta}  =  \frac{\xi_{x}}{J}.
	\end{aligned}
	\label{eq:A1}
\end{equation}
考虑 $(x, y)$ 坐标下的任意守恒型方程组
\begin{equation}
	\frac{\partial U}{\partial t}+\frac{\partial E}{\partial x}+\frac{\partial F}{\partial y}=0,
	\label{eq:A2}
\end{equation}
在坐标变换下有
\begin{equation}
	\begin{aligned}
		\frac{\partial E}{\partial x}=\frac{\partial E}{\partial \xi} \xi_{x}+\frac{\partial E}{\partial \eta} \eta_{x}, \\
		\frac{\partial F}{\partial y}=\frac{\partial F}{\partial \xi} \xi_{y}+\frac{\partial F}{\partial \eta} \eta_{y}.
	\end{aligned}
\end{equation}
代入\cref{eq:A2}可得
\begin{equation}
	\frac{\partial U}{\partial t}+\left(\frac{\partial E}{\partial \xi} \xi_{x}+\frac{\partial F}{\partial \xi} \xi_{y}\right)+\left(\frac{\partial E}{\partial \eta} \eta_{x}+\frac{\partial F}{\partial \eta} \eta_{y}\right)=0
	\label{eq:A3}
\end{equation}
为将上式变为守恒形式,利用\cref{eq:A1}可得
\begin{equation}
	\begin{aligned}
		\frac{\partial}{\partial \xi}\left(\frac{E \xi_{x}+F \xi_{y}}{J}\right)=\frac{1}{J}\left(\frac{\partial E}{\partial \xi} \xi_{x}+\frac{\partial F}{\partial \xi} \xi_{y}\right)+E \frac{\partial}{\partial \xi}\left(\frac{1}{J} \xi_{x}\right)+F \frac{\partial}{\partial \xi}\left(\frac{1}{J} \xi_{y}\right) \\
		\frac{\partial}{\partial \eta}\left(\frac{E \eta_{x}+F \eta_{y}}{J}\right)=\frac{1}{J}\left(\frac{\partial E}{\partial \eta} \eta_{x}+\frac{\partial F}{\partial \eta} \eta_{y}\right)+E \frac{\partial}{\partial \eta}\left(\frac{1}{J} \eta_{x}\right)+F \frac{\partial}{\partial \eta}\left(\frac{1}{J} \eta_{y}\right)
	\end{aligned}
	\label{eq:A4}
\end{equation}
由\cref{eq:A1}可得
\begin{equation}
	\begin{aligned}
		\frac{\partial}{\partial \xi}\left(\frac{1}{J} \xi_{x}\right)+\frac{\partial}{\partial \eta}\left(\frac{1}{J} \eta_{x}\right)=0, \\
		\frac{\partial}{\partial \xi}\left(\frac{1}{J} \xi_{y}\right)+\frac{\partial}{\partial \eta}\left(\frac{1}{J} \eta_{y}\right)=0 .
	\end{aligned}
\end{equation}
将\cref{eq:A4}中的两个式子相加,代入\cref{eq:A3}得到
\begin{equation}
	\frac{\partial}{\partial t}\left(\frac{U}{J}\right)+\frac{\partial}{\partial \xi}\left(\frac{E \xi_{x}+F \xi_{y}}{J}\right)+\frac{\partial}{\partial \eta}\left(\frac{E \eta_{x}+F \eta_{y}}{J}\right)=0
	\label{eq:A5}
\end{equation}
即为 $(\xi, \eta)$ 坐标下守恒形式的方程组。\cite[P28]{cfd}


\subsection{二维可压缩NS方程(守恒形式)}

下面具体考虑二维可压缩NS 方程组。其中,
\begin{equation}
	U=
	\begin{pmatrix}
		\rho   \\
		\rho u \\
		\rho v \\
		\rho \varepsilon
	\end{pmatrix},
	\quad E=
	\begin{pmatrix}
		\rho u                  \\
		\rho u^{2}+p-\tau_{x x} \\
		\rho u v-\tau_{x y}     \\
		(\rho \varepsilon+p) u-u \tau_{x x}-v \tau_{x y}+q_{x}
	\end{pmatrix},\quad
	F=\begin{pmatrix}
		\rho v                  \\
		\rho u v-\tau_{x y}     \\
		\rho v^{2}+p-\tau_{y y} \\
		(\rho \varepsilon+p) v-u \tau_{x y}-v \tau_{y y}+q_{y}
	\end{pmatrix}.
\end{equation}
设
\begin{equation}
	u^{*}=u \xi_{x}+v \xi_{y}
\end{equation}
\begin{equation}
	v^{*}=u \eta_{x}+v \eta_{y} .
\end{equation}

代入\cref{eq:A5}, 可以得到 $(\xi, \eta)$ 坐标下的方程组为
\begin{equation}
	\frac{\partial}{\partial t} \hat{U}+\frac{\partial}{\partial \xi} \hat{E}+\frac{\partial}{\partial \eta} \hat{F}=0,
	\label{eq:A6}
\end{equation}
其中,
\begin{equation}
	\hat{U}=\frac{1}{J}
	\begin{pmatrix}
		\rho   \\
		\rho u \\
		\rho v \\
		\rho \varepsilon
	\end{pmatrix},\quad
	\hat{E}=\frac{1}{J}\begin{pmatrix}
		\rho u^{*}                                                   \\
		\rho u u^{*}+p \xi_{x}-\tau_{x x} \xi_{x}-\tau_{x y} \xi_{y} \\
		\rho v u^{*}+p \xi_{y}-\tau_{x y} \xi_{x}-\tau_{y y} \xi_{y} \\
		(\rho \varepsilon+p) u^{*}-\left(u \tau_{x x}+v \tau_{x y}\right) \xi_{x}-\left(u \tau_{x y}+v \tau_{y y}\right) \xi_{y}+q_{x} \xi_{x}+q_{y} \xi_{y}
	\end{pmatrix},
\end{equation}
\begin{equation}
	\hat{F}=\frac{1}{J}\begin{pmatrix}
		\rho v^{*}                                                      \\
		\rho u v^{*}+p \eta_{x}-\tau_{x x} \eta_{x}-\tau_{x y} \eta_{y} \\
		\rho v v^{*}+p \eta_{y}-\tau_{x y} \eta_{x}-\tau_{y y} \eta_{y} \\
		(\rho \varepsilon+p) v^{*}-\left(u \tau_{x x}+v \tau_{x y}\right) \eta_{x}-\left(u \tau_{x y}+v \tau_{y y}\right) \eta_{y}+q_{x} \eta_{x}+q_{y} \eta_{y}
	\end{pmatrix}.
\end{equation}
\begin{equation}
	\begin{aligned}
		\tau_{x x} & =\frac{2}{3} \mu\left(2 \frac{\partial u}{\partial x}-\frac{\partial v}{\partial y}\right)=\frac{2}{3} \mu\left[2\left(u_{\xi} \xi_{x}+u_{\eta} \eta_{x}\right)-\left(v_{\xi} \xi_{y}+v_{\eta} \eta_{y}\right)\right], \\
		\tau_{x y} & =\tau_{y x}=\mu\left(\frac{\partial u}{\partial y}+\frac{\partial v}{\partial x}\right)=\mu\left(u_{\xi} \xi_{y}+u_{\eta} \eta_{y}+v_{\xi} \xi_{x}+v_{\eta} \eta_{x}\right),                                           \\
		\tau_{y y} & =\frac{2}{3} \mu\left(2 \frac{\partial v}{\partial y}-\frac{\partial u}{\partial x}\right)=\frac{2}{3} \mu\left[2\left(v_{\xi} \xi_{y}+v_{\eta} \eta_{y}\right)-\left(u_{\xi} \xi_{x}+u_{\eta} \eta_{x}\right)\right], \\
		q_{x}      & =-k \frac{\partial T}{\partial x}=-k\left(\frac{\partial T}{\partial \xi} \xi_{x}+\frac{\partial T}{\partial \eta} \eta_{x}\right),                                                                                    \\
		q_{y}      & =-k \frac{\partial T}{\partial y}=-k\left(\frac{\partial T}{\partial \xi} \xi_{y}+\frac{\partial T}{\partial \eta} \eta_{y}\right) .
	\end{aligned}
\end{equation}

\subsection{二维不可压缩NS方程(原始变量形式)}

如果考虑二维不可压缩NS 方程,假设 $\rho$ 为常数,暂时不考虑体力项,当体力有势时,体 力项可以与压力项合并在一起,定义变量
\begin{equation}
	\tilde{p}=\frac{p}{\rho}
\end{equation}

不考虑热传导,\cref{eq:A2}中的各变量为
\begin{equation}
	U=\begin{pmatrix}
		0 \\
		u \\
		v
	\end{pmatrix}, \quad E=\begin{pmatrix}
		u                                   \\
		u^{2}+\tilde{p}-\tau_{x x}^{\prime} \\
		u v-\tau_{x y}^{\prime}
	\end{pmatrix},\quad
	F=\begin{pmatrix}
		v                       \\
		u v-\tau_{x y}^{\prime} \\
		v^{2}+\tilde{p}-\tau_{y y}^{\prime}
	\end{pmatrix}.
\end{equation}
根据\cref{eq:A5}可以得到,坐标 $(\xi, \eta)$ 下的守恒型方程组仍为\cref{eq:A6}的形式,其中
\begin{equation}
	\hat{U}=\frac{1}{J}\begin{pmatrix}
		0 \\
		u \\
		v
	\end{pmatrix},\quad
	\hat{E}=\frac{1}{J}\begin{pmatrix}
		u^{*}                                                                             \\
		u u^{*}+\tilde{p} \xi_{x}-\tau_{x x}^{\prime} \xi_{x}-\tau_{x y}^{\prime} \xi_{y} \\
		v u^{*}+\tilde{p} \xi_{y}-\tau_{x y}^{\prime} \xi_{x}-\tau_{y y}^{\prime} \xi_{y}
	\end{pmatrix}
\end{equation}
\begin{equation}
	\hat{F}=\frac{1}{J}\begin{pmatrix}
		v^{*}                                                                                     \\
		\rho u v^{*}+\tilde{p} \eta_{x}-\tau_{x x}^{\prime} \eta_{x}-\tau_{x y}^{\prime} \eta_{y} \\
		\rho v v^{*}+\tilde{p} \eta_{y}-\tau_{x y}^{\prime} \eta_{x}-\tau_{y y}^{\prime} \eta_{y}
	\end{pmatrix}.
\end{equation}
对于不可压缩流体,
\begin{equation}
	\begin{aligned}
		\tau_{x x}^{\prime} & =2 \nu \frac{\partial u}{\partial x}=2 \nu\left(u_{\xi} \xi_{x}+u_{\eta} \eta_{x}\right),                                                                         \\
		\tau_{x y}^{\prime} & =\nu\left(\frac{\partial u}{\partial y}+\frac{\partial v}{\partial x}\right)=\nu\left(u_{\xi} \xi_{y}+u_{\eta} \eta_{y}+v_{\xi} \xi_{x}+v_{\eta} \eta_{x}\right), \\
		\tau_{y y}^{\prime} & =2 \nu \frac{\partial v}{\partial y}=2 \nu\left(v_{\xi} \xi_{y}+v_{\eta} \eta_{y}\right).
	\end{aligned}
\end{equation}

若考虑非守恒形式的二维可压缩NS 方程组,暂时不考虑能量方程,则有
\begin{equation}
	\left\{\begin{aligned}
		 & \frac{\partial \rho}{\partial t}+u \frac{\partial \rho}{\partial x}+v \frac{\partial \rho}{\partial y}+\rho\left(\frac{\partial u}{\partial x}+\frac{\partial v}{\partial y}\right)=0,                                                    \\
		 & \rho \frac{\partial u}{\partial t}+\rho u \frac{\partial u}{\partial x}+\rho v \frac{\partial u}{\partial y}=-\frac{1}{\rho} \frac{\partial p}{\partial x}+\frac{\partial \tau_{x x}}{\partial x}+\frac{\partial \tau_{x y}}{\partial y}, \\
		 & \rho \frac{\partial v}{\partial t}+\rho u \frac{\partial v}{\partial x}+\rho v \frac{\partial v}{\partial y}=-\frac{1}{\rho} \frac{\partial p}{\partial y}+\frac{\partial \tau_{x y}}{\partial x}+\frac{\partial \tau_{y y}}{\partial y}.
	\end{aligned}\right.
\end{equation}
将坐标变换关系代入可得
\begin{equation}
	\left\{
	\begin{aligned}
		 & \frac{\partial \rho}{\partial t}+u\left(\frac{\partial \rho}{\partial \xi} \xi_{x}+\frac{\partial \rho}{\partial \eta} \eta_{x}\right)+v\left(\frac{\partial \rho}{\partial \xi} \xi_{y}+\frac{\partial \rho}{\partial \eta} \eta_{y}\right)+\rho\left(\frac{\partial u}{\partial \xi} \xi_{x}+\frac{\partial u}{\partial \eta} \eta_{x}+\frac{\partial v}{\partial \xi} \xi_{y}+\frac{\partial v}{\partial \eta} \eta_{y}\right)=0      \\
		 & \rho \frac{\partial u}{\partial t}+\rho u\left(\frac{\partial u}{\partial \xi} \xi_{x}+\frac{\partial u}{\partial \eta} \eta_{x}\right)+\rho v\left(\frac{\partial u}{\partial \xi} \xi_{y}+\frac{\partial u}{\partial \eta} \eta_{y}\right)=-\frac{1}{\rho}\left(\frac{\partial p}{\partial \xi} \xi_{x}+\frac{\partial p}{\partial \eta} \eta_{x}\right)+\frac{\partial \tau_{x x}}{\partial x}+\frac{\partial \tau_{x y}}{\partial y} \\
		 & \rho \frac{\partial v}{\partial t}+\rho u\left(\frac{\partial v}{\partial \xi} \xi_{x}+\frac{\partial v}{\partial \eta} \eta_{x}\right)+\rho v\left(\frac{\partial v}{\partial \xi} \xi_{y}+\frac{\partial v}{\partial \eta} \eta_{y}\right)=-\frac{1}{\rho}\left(\frac{\partial p}{\partial \xi} \xi_{y}+\frac{\partial p}{\partial \eta} \eta_{y}\right)+\frac{\partial \tau_{x y}}{\partial x}+\frac{\partial \tau_{y y}}{\partial y}
	\end{aligned}\right.
\end{equation}
若考虑非守恒形式的二维不可压NS 方程组,并假设 $\nu$ 为常数,则有
\begin{equation}
	\left\{\begin{aligned}
		 & \frac{\partial u}{\partial x}+\frac{\partial v}{\partial y}=0                                                                                                                                                                          \\
		 & \frac{\partial u}{\partial t}+u \frac{\partial u}{\partial x}+v \frac{\partial u}{\partial y}+\frac{1}{\rho} \frac{\partial p}{\partial x}=\nu\left(\frac{\partial^{2} u}{\partial x^{2}}+\frac{\partial^{2} u}{\partial y^{2}}\right) \\
		 & \frac{\partial v}{\partial t}+u \frac{\partial v}{\partial x}+v \frac{\partial v}{\partial y}+\frac{1}{\rho} \frac{\partial p}{\partial y}=\nu\left(\frac{\partial^{2} v}{\partial x^{2}}+\frac{\partial^{2} v}{\partial y^{2}}\right)
	\end{aligned}\right.
\end{equation}

将坐标变换关系代入可以得到
\begin{equation}
	\left\{\begin{aligned}
		 & \frac{\partial u}{\partial \xi} \xi_{x}+\frac{\partial u}{\partial \eta} \eta_{x}+\frac{\partial v}{\partial \xi} \xi_{y}+\frac{\partial v}{\partial \eta} \eta_{y}=0                                                                                                                                                                                                                                                                                                                                                                                    \\
		 & \frac{\partial u}{\partial t}+u\left(\frac{\partial u}{\partial \xi} \xi_{x}+\frac{\partial u}{\partial \eta} \eta_{x}\right)+v\left(\frac{\partial u}{\partial \xi} \xi_{y}+\frac{\partial u}{\partial \eta} \eta_{y}\right)+\frac{1}{\rho}\left(\frac{\partial p}{\partial \xi} \xi_{x}+\frac{\partial p}{\partial \eta} \eta_{x}\right)=                                                                                                                                                                                                              \\
		 & \nu\left[\left(\frac{\partial^{2} u}{\partial \xi^{2}} \xi_{x}^{2}+2 \frac{\partial^{2} u}{\partial \xi \partial \eta} \xi_{x} \eta_{x}+\frac{\partial^{2} u}{\partial \eta^{2}} \eta_{x}^{2}\right)+\left(\frac{\partial u}{\partial \xi} \frac{\partial \xi_{x}}{\partial \xi}+\frac{\partial u}{\partial \eta} \frac{\partial \eta_{x}}{\partial \xi}\right) \xi_{x}+\left(\frac{\partial u}{\partial \xi} \frac{\partial \xi_{\eta}}{\partial \xi}+\frac{\partial u}{\partial \eta} \frac{\partial \eta_{\eta}}{\partial \xi}\right) \eta_{x}\right. \\
		 & \left.+\left(\frac{\partial^{2} u}{\partial \xi^{2}} \xi_{y}^{2}+2 \frac{\partial^{2} u}{\partial \xi \partial \eta} \xi_{y} \eta_{y}+\frac{\partial^{2} u}{\partial \eta^{2}} \eta_{y}^{2}\right)+\left(\frac{\partial u}{\partial \xi} \frac{\partial \xi_{y}}{\partial \xi}+\frac{\partial u}{\partial \eta} \frac{\partial \eta_{y}}{\partial \xi}\right) \xi_{y}+\left(\frac{\partial u}{\partial \xi} \frac{\partial \xi_{\eta}}{\partial \xi}+\frac{\partial u}{\partial \eta} \frac{\partial \eta_{\eta}}{\partial \xi}\right) \eta_{y}\right]   \\
		 & \frac{\partial v}{\partial t}+u\left(\frac{\partial v}{\partial \xi} \xi_{x}+\frac{\partial v}{\partial \eta} \eta_{x}\right)+v\left(\frac{\partial v}{\partial \xi} \xi_{y}+\frac{\partial v}{\partial \eta} \eta_{y}\right)+\frac{1}{\rho}\left(\frac{\partial p}{\partial \xi} \xi_{y}+\frac{\partial p}{\partial \eta} \eta_{y}\right)=                                                                                                                                                                                                              \\
		 & \nu\left[\left(\frac{\partial^{2} u}{\partial \xi^{2}} \xi_{x}^{2}+2 \frac{\partial^{2} u}{\partial \xi \partial \eta} \xi_{x} \eta_{x}+\frac{\partial^{2} u}{\partial \eta^{2}} \eta_{x}^{2}\right)+\left(\frac{\partial u}{\partial \xi} \frac{\partial \xi_{x}}{\partial \xi}+\frac{\partial u}{\partial \eta} \frac{\partial \eta_{x}}{\partial \xi}\right) \xi_{x}+\left(\frac{\partial u}{\partial \xi} \frac{\partial \xi_{\eta}}{\partial \xi}+\frac{\partial u}{\partial \eta} \frac{\partial \eta_{\eta}}{\partial \xi}\right) \eta_{x}\right. \\
		 & \left.+\left(\frac{\partial^{2} v}{\partial \xi^{2}} \xi_{y}^{2}+2 \frac{\partial^{2} v}{\partial \xi \partial \eta} \xi_{y} \eta_{y}+\frac{\partial^{2} v}{\partial \eta^{2}} \eta_{y}^{2}\right)+\left(\frac{\partial v}{\partial \xi} \frac{\partial \xi_{y}}{\partial \xi}+\frac{\partial v}{\partial \eta} \frac{\partial \eta_{y}}{\partial \xi}\right) \xi_{y}+\left(\frac{\partial v}{\partial \xi} \frac{\partial \xi_{\eta}}{\partial \xi}+\frac{\partial v}{\partial \eta} \frac{\partial \eta_{\eta}}{\partial \xi}\right) \eta_{y}\right]
	\end{aligned}\right.
\end{equation}


\subsection{极坐标变换}

考虑极坐标 $(r, \theta)$, 极坐标变换下有如下关系
\begin{gather}
	r_{x}=\frac{x}{\sqrt{x^{2}+y^{2}}}=\cos \theta, \quad r_{y}=\frac{y}{\sqrt{x^{2}+y^{2}}}=\sin \theta,                   \\
	\theta_{x}=-\frac{y}{x^{2}+y^{2}}=-\frac{\sin \theta}{r}, \quad \theta_{y}=\frac{x}{x^{2}+y^{2}}=\frac{\cos \theta}{r}, \\
	J=\frac{1}{r} .
\end{gather}

将上述关系代入\cref{eq:A5}可得到极坐标下二维可压缩NS 方程的守恒形式
\begin{equation}
	\hat{U}=r\left(\begin{aligned}
			\rho   \\
			\rho u \\
			\rho v \\
			\rho \varepsilon
		\end{aligned}\right),
\end{equation}

\begin{gather}
	\hat{E}=r \begin{pmatrix}
		\rho u^{*}                                                               \\
		\rho u u^{*}+p \cos \theta-\tau_{x x} \cos \theta-\tau_{x y} \sin \theta \\
		\rho v u^{*}+p \sin \theta-\tau_{x y} \cos \theta-\tau_{y y} \sin \theta \\
		(\rho \varepsilon+p) u^{*}-\left(u \tau_{x x}+v \tau_{x y}\right) \cos \theta-\left(u \tau_{x y}+v \tau_{y y}\right) \sin \theta-k T_{\xi}
	\end{pmatrix},\\
	\hat{F}=r\begin{pmatrix}
		\rho v^{*}                                                               \\
		\rho u v^{*}-p \sin \theta+\tau_{x x} \sin \theta-\tau_{x y} \cos \theta \\
		\rho v v^{*}+p \cos \theta+\tau_{x y} \sin \theta-\tau_{y y} \cos \theta \\
		(\rho \varepsilon+p) v^{*}-\left(u \tau_{x x}-v \tau_{x y}\right) \sin \theta-\left(u \tau_{x y}+v \tau_{y y}\right) \cos \theta-k T_{\eta}
	\end{pmatrix}.
\end{gather}
其中,
\begin{gather}
	u^{*}=u \cos \theta+v \sin \theta,                                                                                                                                                                                                                                       \\
	v^{*}=-u \sin \theta+v \cos \theta,                                                                                                                                                                                                                                      \\
	\tau_{x x}=\frac{2}{3} \mu\left[2\left(\frac{\partial u}{\partial r} \cos \theta-\frac{\partial u}{\partial \theta} \frac{\sin \theta}{r}\right)-\left(\frac{\partial v}{\partial r} \sin \theta+\frac{\partial v}{\partial \theta} \frac{\cos \theta}{r}\right)\right], \\
	\tau_{x y}=\mu\left[\frac{\partial u}{\partial r} \sin \theta+\frac{\partial u}{\partial \theta} \frac{\cos \theta}{r}+\frac{\partial v}{\partial r} \cos \theta-\frac{\partial v}{\partial \theta} \frac{\sin \theta}{r}\right],                                        \\
	\tau_{y y}=\frac{2}{3} \mu\left[2\left(\frac{\partial v}{\partial r} \sin \theta+\frac{\partial v}{\partial \theta} \frac{\cos \theta}{r}\right)-\left(\frac{\partial u}{\partial r} \cos \theta-\frac{\partial u}{\partial \theta} \frac{\sin \theta}{r}\right)\right].
\end{gather}
对于二维不可压NS 方程,可得到其守恒形式为
\begin{gather}
	\hat{U}=r \begin{pmatrix}
		0 \\
		u \\
		v
	\end{pmatrix}, \\
	\hat{E}=r \begin{pmatrix}
		u^{*}                                                                                         \\
		u u^{*}+\tilde{p} \cos \theta-\tau_{x x}^{\prime} \cos \theta-\tau_{x y}^{\prime} \sin \theta \\
		v u^{*}+\tilde{p} \sin \theta-\tau_{x y}^{\prime} \cos \theta-\tau_{y y}^{\prime} \sin \theta
	\end{pmatrix},\\
	\hat{F}= r \begin{pmatrix}
		v^{*}                                                                                         \\
		u v^{*}-\tilde{p} \sin \theta+\tau_{x x}^{\prime} \sin \theta-\tau_{x y}^{\prime} \cos \theta \\
		v v^{*}+\tilde{p} \cos \theta+\tau_{x y}^{\prime} \sin \theta-\tau_{y y}^{\prime} \cos \theta
	\end{pmatrix}.
\end{gather}
其中,
\begin{equation}
	u^{*}=u \cos \theta+v \sin \theta,
\end{equation}
\begin{gather}
	v^{*}=-u \sin \theta+v \cos \theta,                                                                                                                                                                                                        \\
	\tau_{x x}^{\prime}=2 \nu\left(\frac{\partial u}{\partial r} \cos \theta-\frac{\partial u}{\partial \theta} \frac{\sin \theta}{r}\right),                                                                                                \\
	\tau_{x y}^{\prime}=\nu\left(\frac{\partial u}{\partial r} \sin \theta+\frac{\partial u}{\partial \theta} \frac{\cos \theta}{r}+\frac{\partial v}{\partial r} \cos \theta-\frac{\partial v}{\partial \theta} \frac{\sin \theta}{r}\right), \\
	\tau_{y y}^{\prime}=2 \nu\left(\frac{\partial v}{\partial r} \sin \theta+\frac{\partial v}{\partial \theta} \frac{\cos \theta}{r}\right).
\end{gather}
% 上述讨论得到的结果是在 $(x, y)$ 坐标下的分量 $(u, v)$ 得到的,如果要以 $(\xi, \eta)$ 坐标下的分 量 $\left(u_{r}, u_{\theta}\right) \quad($ 此处下标不表示求导) 表示坐标变换后的结果,可借助速度分量之间的关系获 得
为了得到极坐标下的速度分量$\left(u_{r}, u_{\theta}\right)$的方程,做替换
\begin{gather}
	u=u_{r} \cos \theta-u_{\theta} \sin \theta, \\
	v=u_{r} \sin \theta+u_{\theta} \cos \theta .
\end{gather}
% 为得到 $r, \theta$ 方向的动量方程,需做如下计算:
% \begin{gather}
% 	x \text { 分量 } \times \cos \theta+y \text { 分量 } \times \sin \theta=r \text { 分量, } \\
% 	-x \text { 分量 } \times \sin \theta+y \text { 分量 } \times \cos \theta=\theta \text { 分量. }
% \end{gather}
% 此时得到的结果是非守恒形式的。
最终可以得到
\begin{equation}
	\left\{\begin{aligned}
		 & \frac{\partial\left(r u_{r}\right)}{\partial r}+\frac{\partial u_{\theta}}{\partial \theta}=0, \\ &\frac{\partial u_{r}}{\partial t}+u_{r} \frac{\partial u_{r}}{\partial r}+\frac{u_{\theta}}{r} \frac{\partial u_{r}}{\partial \theta}-\frac{u_{\theta}^{2}}{r}=-\frac{1}{\rho} \frac{\partial p}{\partial r}+\nu\left[\frac{1}{r} \frac{\partial}{\partial r}\left(r \frac{\partial u_{r}}{\partial r}\right)-\frac{u_{r}}{r^{2}}+\frac{1}{r^{2}} \frac{\partial^{2} u_{r}}{\partial \theta^{2}}-\frac{2}{r^{2}} \frac{\partial u_{\theta}}{\partial \theta}\right],
		\\ &\frac{\partial u_{\theta}}{\partial t}+u_{r} \frac{\partial u_{\theta}}{\partial r}+\frac{u_{\theta}}{r} \frac{\partial u_{\theta}}{\partial \theta}+\frac{u_{r} u_{\theta}}{r}=-\frac{1}{\rho r} \frac{\partial p}{\partial \theta}+\nu\left[\frac{1}{r} \frac{\partial}{\partial r}\left(r \frac{\partial u_{\theta}}{\partial r}\right)-\frac{u_{\theta}}{r^{2}}+\frac{1}{r^{2}} \frac{\partial^{2} u_{\theta}}{\partial \theta^{2}}+\frac{2}{r^{2}} \frac{\partial u_{r}}{\partial \theta}\right].\end{aligned}\right.
\end{equation}

对于可压缩NS方程,利用拉梅系数,通过分析各算子在曲线坐标下的形式可得\cite[P190]{wuwangyi}
\begin{equation}
	\left\{\begin{aligned}
		 & \frac{\partial \rho}{\partial t}+\frac{1}{r} \frac{\partial\left(\rho r u_{r}\right)}{\partial r}+\frac{1}{r} \frac{\partial\left(\rho u_{\theta}\right)}{\partial \theta}=0,                                     \\
		 & \rho\left(\frac{\Dif u_{r}}{\Dif t}-\frac{u_{\theta}^{2}}{r}\right)=\frac{1}{r}\left[\frac{\partial\left(r P_{r r}\right)}{\partial r}+\frac{P_{r \theta}}{\partial \theta}-P_{\theta \theta}\right],             \\
		 & \rho\left(\frac{\Dif u_{\theta}}{\Dif t}+\frac{u_{r} u_{\theta}}{r}\right)=\frac{1}{r}\left[\frac{\partial\left(r P_{r \theta}\right)}{\partial r}+\frac{P_{\theta \theta}}{\partial \theta}+P_{r \theta}\right].
	\end{aligned}\right.
\end{equation}
其中
\begin{equation}
	\begin{aligned}
		P_{r r}             & =-p+2 \mu\left(\frac{\partial u_{r}}{\partial r}-\frac{1}{3} \nabla \cdot \bm{u}\right)                                         \\
		P_{\theta \theta}   & =-p+2 \mu\left(\frac{1}{r} \frac{\partial u_{\theta}}{\partial \theta}+\frac{u_{r}}{r}-\frac{1}{3} \nabla \cdot \bm{u}\right)   \\
		P_{r \theta}        & =\mu\left(\frac{\partial u_{\theta}}{\partial r}+\frac{1}{r} \frac{\partial u_{r}}{\partial \theta}-\frac{u_{\theta}}{r}\right) \\
		\frac{\Dif}{\Dif t} & =\frac{\partial}{\partial t}+u_{r} \frac{\partial}{\partial r}+\frac{u_{\theta}}{r} \frac{\partial}{\partial \theta}            \\
		\nabla \cdot \bm{u} & =\frac{1}{r}\left[\frac{\partial\left(r u_{r}\right)}{\partial r}+\frac{\partial u_{\theta}}{\partial \theta}\right]
	\end{aligned}
\end{equation}

\subsection{边界条件}
对于无穷远边界条件
\begin{equation}
	r \rightarrow \infty, \quad \bm{u}=\bm{u}_{\infty}, \quad \rho=\rho_{\infty}, \quad p=p_{\infty}
\end{equation}
对于一般的边界条件,例如对于某一个面,法向为$\bm{n}$,切向为$\bm{\tau}$则把速度投影到法向和切向
\begin{equation}
	\bm{u}_{n} = \bm{n}\cdot \bm{u},\quad \bm{u}_{\tau} = \bm{\tau}\cdot \bm{u}.
\end{equation}
然后按照无滑移或无穿透边界条件给定边界条件。


\section{B}

给定三维不可 压Navier-Stokes(INS)方程 的原始变量形 式的定
解问题(讲义“CFDLect08-incom_cn.pdf"中第7页),引入向量势
函数 $\bm{A}$ 和涡向量函数 $\bm{\omega}$, 推导三维INS方程的涡向量势公式(讲义“CFDLect09-incom_cn.pdf"中第35页)。

三维INS 方程的原始变量形式定解问题为
\begin{equation}
	\left\{\begin{aligned}
		 & \frac{\partial \bm{u}}{\partial t}+(\bm{u} \cdot \nabla) \bm{u}=-\nabla \frac{p}{\rho}+\nu \nabla^{2} \bm{u}+\frac{\bm{f}_{B}}{\rho}, \\
		 & \nabla \cdot \bm{u}=0,                                                                                                                \\
		 & \left.\bm{u}\right|_{\partial \Omega}=\bm{u}_{b}, \quad \bm{u}(\bm{x}, 0)=\bm{u}_{0} .
	\end{aligned}\right.
\end{equation}
上式中已经考虑 $\rho$ 为常数。引入向量势函数 $\bm{A}$ 和涡向量函数 $\bm{\omega}$
\begin{equation}
	\bm{u}=\nabla \times \bm{A}, \quad \bm{\omega}=\nabla \times \bm{u}.
\end{equation}
利用关系式
\begin{equation}
	(\bm{u} \cdot \nabla) \bm{u}=\nabla \frac{|\bm{u}|^{2}}{2}+\bm{\omega} \times \bm{u},
\end{equation}
三维INS 方程组的动量方程可以化为兰姆-葛罗米柯形式
\begin{equation}
	\frac{\partial \bm{u}}{\partial t}+\nabla \frac{|\bm{u}|^{2}}{2}+\bm{\omega} \times \bm{u}=-\nabla \frac{p}{\rho}+\nu \nabla^{2} \bm{u}+\frac{\bm{f}_{B}}{\rho}
\end{equation}
对上式取旋度,假设体力 $\bm{f}_{B}$ 有势,并利用
\begin{equation}
	\nabla \times \nabla^{2} \bm{u}=\nabla^{2} \bm{\omega},
\end{equation}
则
% 下面利用张量形式进行证明:
% (i)
% \begin{equation}
% 	\begin{aligned}
% 		\nabla \frac{|\bm{u}|^{2}}{2}+\bm{\omega} \times \bm{u} & =\frac{1}{2} \frac{\partial u_{i} u_{i}}{\partial x_{j}}+\varepsilon_{j k l}\left(\varepsilon_{k m n} \frac{\partial u_{n}}{\partial x_{m}}\right) u_{l}  \\
% 		                                                        & =u_{i} \frac{\partial u_{i}}{\partial x_{j}}+\left(\delta_{l m} \delta_{j n}-\delta_{l n} \delta_{j m}\right) u_{l} \frac{\partial u_{n}}{\partial x_{m}} \\
% 		                                                        & =u_{i} \frac{\partial u_{i}}{\partial x_{j}}+u_{l} \frac{\partial u_{j}}{\partial x_{l}}-u_{l} \frac{\partial u_{l}}{\partial x_{j}}                      \\
% 		                                                        & =u_{l} \frac{\partial u_{j}}{\partial x_{l}}=(\bm{u} \cdot \nabla) \bm{u}
% 	\end{aligned}
% \end{equation}

% (ii)
% \begin{equation}
% 	\begin{aligned}
% 		\nabla \times \nabla^{2} \bm{u} & =\varepsilon_{i j k} \frac{\partial}{\partial x_{j}}\left(\frac{\partial^{2} u_{k}}{\partial x_{l} \partial x_{l}}\right)                        \\
% 		                                & =\frac{\partial^{2}}{\partial x_{l} \partial x_{l}}\left(\varepsilon_{i j k} \frac{\partial u_{k}}{\partial x_{j}}\right)=\nabla^{2} \bm{\omega}
% 	\end{aligned}
% \end{equation}
% 对式(21) 两边取旋度,并考虑体力 $\bm{f}_{B}$ 有势,得到
\begin{equation}
	\frac{\partial \bm{\omega}}{\partial t}+\nabla \times(\bm{\omega} \times \bm{u})=\nu \nabla^{2} \bm{\omega} .
\end{equation}

三维INS方程的涡向量势公式
\begin{equation}
	\left\{\begin{aligned}
		 & \frac{\partial \bm{\omega}}{\partial t}+\nabla \times(\bm{\omega} \times \bm{u})=\nu \nabla^{2} \bm{\omega}, \\
		 & \bm{\omega}=\nabla \times \bm{u}, \quad \nabla \cdot \bm{u}=0,                                               \\
		 & \left.\bm{u}\right|_{\partial \Omega}=\bm{u}_{b}, \quad \bm{u}(\bm{x}, 0)=\bm{u}_{0}.
	\end{aligned}\right.
\end{equation}


\section{C}

考虑一维网格生成问题. 设逻辑区域(也称为参考区域) $\Omega_{c}=\{\xi: 0 \leq \xi \leq 1\}$ 到物理区域 $\Omega_{p}=\{x: a \leq x \leq b\}$ 的坐标变换 $\xi=\xi(x)$ 满足:
\begin{equation}
	\xi_{x x}=P,
	\label{eq:C1}
\end{equation}
$  P$ 为常数.
分析: 右端项 $P$ 对生成物理区域 $\Omega_{p}$ 的网格的影响.

若$P=0$,则网格是均匀的。对\cref{eq:C1}积分一次可得
\begin{equation}
	\xi_{x}=C,
\end{equation}
其中$C$是常数,且不等于0,否则无法划分网格。再积分一次得
\begin{equation}
	\xi = Cx+D,
\end{equation}
其中$D$也为常数。根据边界条件,端点处重合,所以
\begin{equation}
	\begin{cases}
		C\times a+D=0, \\
		C\times b+D=1.
	\end{cases}\text{or} \quad
	\begin{cases}
		C\times b+D=0, \\
		C\times a+D=1.
	\end{cases}
\end{equation}
解得
\begin{equation}
	\begin{cases}
		D=-\frac{a}{b-a}, \\
		C=\frac{1}{b-a}.
	\end{cases}\text{or} \quad
	\begin{cases}
		D=\frac{b}{b-a}, \\
		C=\frac{1}{a-b}.
	\end{cases}
\end{equation}
若$P$不为0,则两次积分后是二次函数
\begin{equation}
	\xi = \frac{1}{2} Px^2 + Cx + D.
	\label{eq:C2}
\end{equation}
必须要求此二次函数是单调的,否则会出现一个$x$点对应不同的$\xi$.极值点为$x_0=-\frac{C}{P}$,
\begin{equation}
	x_0<a,\quad \text{or}\quad  x_0>b.
\end{equation}
然后要满足在端点处重合
\begin{equation}
	\begin{cases}
		\frac{1}{2}Pa^2 + C\times a+D=0, \\
		\frac{1}{2}Pb^2 + C\times b+D=1.
	\end{cases}\text{or} \quad
	\begin{cases}
		\frac{1}{2}Pb^2 + C\times b+D=0, \\
		\frac{1}{2}Pa^2 + C\times a+D=1.
	\end{cases}
\end{equation}

$P$越大,\cref{eq:C2}的非线性性越强,网格越不均匀,反之, $P$越小,则越接近$P=0$的线性情况。另外,$P$的正负将影响到网格是在哪边更密,哪边更稀疏。



\nocite{*}

\bibliographystyle{plain}

\phantomsection

\addcontentsline{toc}{section}{参考文献} %向目录中添加条目,以章的名义
\bibliography{homework}

\end{document}

