\documentclass[12pt]{article}
\input{/Users/circle/Documents/博一下/homework/setting.tex}
\setcounter{secnumdepth}{2}
\usepackage{autobreak}
\usepackage{amsmath}
\setlength{\parindent}{2em}
\graphicspath{{../}}
\ziju{0.1pt}


%pdf文件设置
\hypersetup{
	pdfauthor={袁磊祺},
	pdftitle={计算流体力学作业3}
}

\title{
		\vspace{-1in} 	
		\usefont{OT1}{bch}{b}{n}
		\normalfont \normalsize \textsc{\LARGE Peking University}\\[0.2cm] % Name of your university/college \\ [25pt]
		\horrule{0.5pt} \\[0.2cm]
		\huge \bfseries{计算流体力学作业3} \\[-0.2cm]
		\horrule{2pt} \\[0.2cm]
}
\author{
		\normalfont 								\normalsize
		College of Engineering \quad 2001111690  \quad 袁磊祺\\	\normalsize
        \today
}
\date{}

\begin{document}

%%%%%%%%%%%%%%%%%%%%%%%%%%%%%%%%%%%%%%%%%%%%%%
\captionsetup[figure]{name={图},labelsep=period}
\captionsetup[table]{name={表},labelsep=period}
\renewcommand\contentsname{目录}
\renewcommand\listfigurename{插图目录}
\renewcommand\listtablename{表格目录}
\renewcommand\refname{参考文献}
\renewcommand\indexname{索引}
\renewcommand\figurename{图}
\renewcommand\tablename{表}
\renewcommand\abstractname{摘\quad 要}
\renewcommand\partname{部分}
\renewcommand\appendixname{附录}
\def\equationautorefname{式}%
\def\footnoteautorefname{脚注}%
\def\itemautorefname{项}%
\def\figureautorefname{图}%
\def\tableautorefname{表}%
\def\partautorefname{篇}%
\def\appendixautorefname{附录}%
\def\chapterautorefname{章}%
\def\sectionautorefname{节}%
\def\subsectionautorefname{小小节}%
\def\subsubsectionautorefname{subsubsection}%
\def\paragraphautorefname{段落}%
\def\subparagraphautorefname{子段落}%
\def\FancyVerbLineautorefname{行}%
\def\theoremautorefname{定理}%
\crefname{figure}{图}{图}
\crefname{equation}{式}{式}
\crefname{table}{表}{表}
%%%%%%%%%%%%%%%%%%%%%%%%%%%%%%%%%%%%%%%%%%%

\maketitle

\section{Cauchy问题的解}

利用特征线理论分析问题
\begin{equation}
	\begin{cases}
		u_{t}+a(u) u_{x}=0, & x \in \mathbb{R},\ t>0, \\
		u(x, 0)=u_{0}(x), & x \in \mathbb{R}.
	\end{cases}
	\label{eq:10}
\end{equation}
并给出(光滑的)解.

{\bfseries 解:}
$u_{t}+a(u) u_{x}=0, u(x, 0)=u_{0}(x) .$ 问题转化为
\begin{equation}
	\begin{cases}
		\frac{\dif x}{\dif t}=a(u), \\
		x(0)=x_{0}.
	\end{cases}
	\quad
	\begin{cases}
		\frac{\dif u}{\dif t}=0, \\
		u(0)=u_{0}\left(x_{0}\right).		
	\end{cases}
\end{equation}

由上述ODE初值问题得
\begin{equation}
	u(x,t) = u_0(x_0),
	\label{eq:11}
\end{equation}
\begin{equation}
	x=x_{0}+a(u_0(x_0))t=x_{0}+a(u(x,t))t,
	\label{eq:111}
\end{equation}
$x_0$依赖于给定的点$(x,t)$,
\begin{equation}
	u(x, t)=u_{0}\left(x-a(u_0(x_0))t\right)=u_{0}\left(x-a(u_0(x_0(x,t)))t\right)=u_{0}\left(x-a(u(x,t))t\right).
	\label{eq:12}
\end{equation}

由\cref{eq:10}得
\begin{equation}
	u_{t}=u_{0}^{\prime}\left(x_{0}\right) \frac{\partial x_{0}}{\partial t}, \quad u_{x}=u_{0}^{\prime}\left(x_{0}\right) \frac{\partial x_{0}}{\partial x}
	\label{eq:121}
\end{equation}

将\cref{eq:111} 的第一等号两端分别对$t$和 $x$ 求导, 得

\begin{gather}
a\left(u_{0}\left(x_{0}\right)\right)+\left[1+a^{\prime} \cdot u_{0}^{\prime}\left(x_{0}\right) \cdot t\right] \frac{\partial x_{0}}{\partial t}=0 
\label{eq:130}
\\
\left(1+a^{\prime} u_{0}^{\prime}\left(x_{0}\right) t\right) \frac{\partial x_{0}}{\partial x}=1
\label{eq:13}
\end{gather}

从\cref{eq:130,eq:13}得
\begin{equation}
	\frac{\partial x_{0}}{\partial t}=-\frac{a\left(u_{0}\left(x_{0}\right)\right)}{1+\left(a^{\prime} u_{0}^{\prime}\right)_{x_{0}} t}, \quad \frac{\partial x_{0}}{\partial x}=\frac{1}{1+\left(a^{\prime} u_{0}^{\prime}\right)_{x_{0}} t}.
\end{equation}

将其代入\cref{eq:121}知
\begin{equation}
	u_{t}+a(u) u_{x}=0.
\end{equation}

$t=0$时
\begin{equation}
	u(0)=u_{0}\left(x_{0}\right),
\end{equation}
所以\cref{eq:12}满足\cref{eq:10}.

\section{无黏 Burgers 方程的定解问题}

\begin{equation}
	\begin{cases}
		u_t+(0.5 u^2)_x=0, \\
		u_0(x) = \cos(\pi x),\quad x \in [-1,1].
	\end{cases}
\end{equation}


此时$a(u)=u$,爆破点
\begin{equation}
	t^* = - \frac{1}{a'u_0'} = \frac{1}{\pi \sin(\pi x_0)},
\end{equation}
只在$x>0$的部分会出现爆破,最快达到爆破的点为$x_0=0.5$,经历时间$t_0^*=\frac{1}{\pi}$.


\section{Burgers 方程 Riemann 问题的弱解}

弱解满足的方程为
\begin{equation}
	\int_{0}^{+\infty} \int_{-\infty}^{+\infty}\left[\phi_{t} \boldsymbol{U}+\phi_{x} \boldsymbol{F}(\boldsymbol{U})\right] \dif x \dif t=-\int_{-\infty}^{+\infty} \phi(x, 0) \boldsymbol{U}(x, 0) \dif x.
\end{equation}

对Burgers 方程有
\begin{equation}
	\int_{0}^{+\infty} \int_{-\infty}^{+\infty}\left[\phi_{t} {u}+\phi_{x} \frac{1}{2}u^2\right] \dif x \dif t=-\int_{-\infty}^{+\infty} \phi(x, 0) {u}(x, 0) \dif x.
\end{equation}

\subsection{激波弱解}


\begin{equation}
	u(x, t)=
	\begin{cases}
		u_{L}, & x<s t, \\
		u_{R}, & x>s t	.
	\end{cases}
\end{equation}
其中
\begin{equation}
	s=\frac{u_L+u_R}{2}.
\end{equation}

\begin{align}
	& \int_{0}^{+\infty} \int_{-\infty}^{+\infty}\left[\phi_{t} {u}+\phi_{x} \frac{1}{2}u^2\right] \dif x \dif t \\
	= & \int_{0}^{+\infty} \int_{-\infty}^{+\infty}\phi_{t} {u} \dif x \dif t + \int_{0}^{+\infty} \int_{-\infty}^{+\infty}\phi_{x} \frac{1}{2}u^2 \dif x \dif t \\
	= & -\int_{-\infty}^{+\infty} \phi(x, 0) {u}(x, 0) \dif x +  \int_{0}^{+\infty} \phi(x, x/s) \left(u_L-u_R\right) \dif x\\
	& + \frac{1}{2} \int_{0}^{+\infty} \phi(st, t) \left(u_L^2-u_R^2\right) \dif t\\
	= & -\int_{-\infty}^{+\infty} \phi(x, 0) {u}(x, 0) \dif x +  s \int_{0}^{+\infty} \phi(x, x/s) \left(u_L-u_R\right) \dif\, (x/s)\\
	& + \frac{1}{2} \int_{0}^{+\infty} \phi(st, t) \left(u_L^2-u_R^2\right) \dif t\\
	= & -\int_{-\infty}^{+\infty} \phi(x, 0) {u}(x, 0) \dif x.
\end{align}

\subsection{$u_L<u_R$弱解}


\begin{equation}
	u(x, t)=
	\begin{cases}
		u_{L}, & x<s t, \\
		u_{R}, & x>s t	.
	\end{cases}
\end{equation}
其中
\begin{equation}
	s=\frac{u_L+u_R}{2}.
\end{equation}

同样的
\begin{align}
	& \int_{0}^{+\infty} \int_{-\infty}^{+\infty}\left[\phi_{t} {u}+\phi_{x} \frac{1}{2}u^2\right] \dif x \dif t \\
	= & \int_{0}^{+\infty} \int_{-\infty}^{+\infty}\phi_{t} {u} \dif x \dif t + \int_{0}^{+\infty} \int_{-\infty}^{+\infty}\phi_{x} \frac{1}{2}u^2 \dif x \dif t \\
	= & -\int_{-\infty}^{+\infty} \phi(x, 0) {u}(x, 0) \dif x +  \int_{0}^{+\infty} \phi(x, x/s) \left(u_L-u_R\right) \dif x\\
	& + \frac{1}{2} \int_{0}^{+\infty} \phi(st, t) \left(u_L^2-u_R^2\right) \dif t\\
	= & -\int_{-\infty}^{+\infty} \phi(x, 0) {u}(x, 0) \dif x +  s \int_{0}^{+\infty} \phi(x, x/s) \left(u_L-u_R\right) \dif\, (x/s)\\
	& + \frac{1}{2} \int_{0}^{+\infty} \phi(st, t) \left(u_L^2-u_R^2\right) \dif t\\
	= & -\int_{-\infty}^{+\infty} \phi(x, 0) {u}(x, 0) \dif x.
\end{align}

\subsection{稀疏波弱解}

\begin{equation}
	u(x, t)=
	\begin{cases}
		u_{L}, & x<s_{m} t \\
		u_{m}, & s_{m} t \leq x \leq u_{m} t \\
		\frac{x}{t}, & u_{m} t \leq x \leq u_{R} t \\
		u_{R}, & x>u_{R} t
	\end{cases}
\end{equation}
也是一个弱解, 其中 $u_{m} \in\left[u_{L}, u_{R}\right]$ 为任意, $s_{m}=\frac{u_{L}+u_{m}}{2} .$

\begin{align}
	& \int_{0}^{+\infty} \int_{-\infty}^{+\infty}\left[\phi_{t} {u}+\phi_{x} \frac{1}{2}u^2\right] \dif x \dif t \\
	= & -\int_{-\infty}^{+\infty} \phi(x, 0) {u}(x, 0) \dif x + \int_{0}^{+\infty} \left( - u_{L} \phi\left(x, \frac{x}{s_{m}}\right)+u_{m} \phi\left(x, \frac{x}{s_{m}}\right)\right. \\
	&\left.-u_{m} \phi\left(x, \frac{x}{u_{m}}\right)+\int_{\frac{x}{u_{R}}}^{\frac{x}{u_m}} \phi_{t} \frac{x}{t} \dif t + u_{R} \phi\left(x, \frac{x}{u_{R}}\right) \right) \dif x\\
	& + \int_{0}^{+\infty}-\frac{1}{2} \left( u_{R}^{2} \phi\left(u_{R} t, t\right)+\int_{u_{m} t}^{u_{R} t}  \frac{x^{2}}{t^{2}} \phi_{x}(x, t) \dif x +  u_{m}^{2} \phi\left(u_{m} t, t\right) + \left(u_{L}^{2}- u_{m}^{2}\right)  \phi\left(s_{m} t, t\right) \right) \dif t.\\
	= & -\int_{-\infty}^{+\infty} \phi(x, 0) {u}(x, 0) \dif x + \int_{0}^{+\infty} \left(-u_{m} \phi\left(x, \frac{x}{u_{m}}\right)+\int_{\frac{x}{u_{R}}}^{\frac{x}{u_m}} \phi_{t} \frac{x}{t} \dif t + u_{R} \phi\left(x, \frac{x}{u_{R}}\right) \right) \dif x\\
	& + \int_{0}^{+\infty}-\frac{1}{2} \left( u_{R}^{2} \phi\left(u_{R} t, t\right)+\int_{u_{m} t}^{u_{R} t}  \frac{x^{2}}{t^{2}} \phi_{x}(x, t) \dif x +  u_{m}^{2} \phi\left(u_{m} t, t\right)  \right) \dif t.
\end{align}






% \nocite{*}

\bibliographystyle{plain}

\phantomsection

\addcontentsline{toc}{section}{参考文献} %向目录中添加条目,以章的名义
\bibliography{homework}

\end{document}
