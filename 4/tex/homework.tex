\documentclass[12pt]{article}
\input{/Users/circle/Documents/博一下/homework/setting.tex}
\setcounter{secnumdepth}{2}
\usepackage{autobreak}
\usepackage{amsmath}
\setlength{\parindent}{2em}
\graphicspath{{../}}
\ziju{0.1pt}


%pdf文件设置
\hypersetup{
	pdfauthor={袁磊祺},
	pdftitle={计算流体力学作业4}
}

\title{
		\vspace{-1in} 	
		\usefont{OT1}{bch}{b}{n}
		\normalfont \normalsize \textsc{\LARGE Peking University}\\[0.2cm] % Name of your university/college \\ [25pt]
		\horrule{0.5pt} \\[0.2cm]
		\huge \bfseries{计算流体力学作业4} \\[-0.2cm]
		\horrule{2pt} \\[0.2cm]
}
\author{
		\normalfont 								\normalsize
		College of Engineering \quad 2001111690  \quad 袁磊祺\\	\normalsize
        \today
}
\date{}

\begin{document}

\input{setc.tex}

\maketitle

\section{1}

\subsection{1a}



完全气体一维守恒形式的Euler方程组为
\begin{equation}
	\frac{\partial}{\partial t}\left(\begin{array}{c}
	\rho \\
	\rho u \\
	E
	\end{array}\right)+\frac{\partial}{\partial x}\left(\begin{array}{c}
	\rho u \\
	\rho u^{2}+p \\
	u(E+p)
	\end{array}\right)=0
	\label{eq:11}
\end{equation}
和 $p=(\gamma-1)\left(E-\frac{1}{2} \rho u^{2}\right), \gamma=C_{p} / C_{v}$ 为常数, $E=\rho e+\frac{1}{2} \rho u^{2}, p=(\gamma-1) \rho e . $ 声速 $a=\sqrt{\gamma p / \rho} .$

假设
\begin{equation}
	x=\rho,\quad y=\rho u, \quad z=E.
	\label{eq:10}
\end{equation}

Jacobi矩阵
\begin{equation}
	\frac{\partial \bm{F}}{\partial \bm{U}}=\boldsymbol{A}(\boldsymbol{U})=\left(\begin{array}{ccc}
	0 & 1 & 0 \\
	-\frac{3y^2}{2x^2} + \gamma \frac{y^2}{2x^2} & \frac{3y}{x} - \frac{\gamma y}{x} & \gamma-1 \\
	(\gamma-1)\frac{y^3}{x^3} -\frac{y}{x^2}\gamma z & \frac{\gamma z}{x} - \frac{(\gamma -1)3y^2}{2x^2} & \gamma \frac{y}{x}
	\end{array}\right),
\end{equation}
再代入\cref{eq:10}得
\begin{equation}
	\boldsymbol{A}(\boldsymbol{U})=\left(\begin{array}{ccc}
	0 & 1 & 0 \\
	\frac{\gamma-3}{2} u^{2} & (3-\gamma) u & \gamma-1 \\
	\frac{\gamma-2}{2} u^{3}-\frac{a^{2} u}{\gamma-1} & \frac{3-2 \gamma}{2} u^{2}+\frac{a^{2}}{\gamma-1} & \gamma u
	\end{array}\right),
\end{equation}
根据
\begin{equation}
	\det{\abs{A-\lambda \bm{I}}} = 0,
\end{equation}
可求出特征值
\begin{equation}
	\lambda_{1} = {u}-a, \quad \lambda_{2}=u, \quad \lambda_{3}=u+a,
\end{equation}
设总焓$H=(E+p) / \rho,$
\begin{equation}
	\begin{array}{c}
	\boldsymbol{R}=\left(\boldsymbol{R}^{(1)}, \boldsymbol{R}^{(2)}, \boldsymbol{R}^{(3)}\right)=\left(\begin{array}{ccc}
	1 & 1 & 1 \\
	u-a & u & u+a \\
	H-u a & \frac{1}{2} u^{2} & H+u a
	\end{array}\right).
	\end{array}
\end{equation}

\begin{align}
	&\nabla_{\boldsymbol{U}} \lambda_{1}(\boldsymbol{U}) \cdot \boldsymbol{R}^{(1)}(\boldsymbol{U}) \\
	=& \left(-\frac{u}{\rho}-\frac{\gamma}{2a} \frac{(\gamma-1)u^2\rho/2-p}{\rho^2},\frac{1}{\rho}+\frac{\gamma}{2a}\frac{(\gamma-1)u}{\rho},-\frac{\gamma}{2a}\frac{(\gamma-1)}{\rho}\right)\cdot \begin{pmatrix}
		1\\u-a\\H-ua
	\end{pmatrix}\\
	=&-\frac{\gamma-1}{\rho^2}p-\frac{a}{\rho}<0,
\end{align}
所以第一个特征场是真正非线性的。

\begin{align}
	&\nabla_{\boldsymbol{U}} \lambda_{2}(\boldsymbol{U}) \cdot \boldsymbol{R}^{(2)}(\boldsymbol{U}) \\
	=& \left(-\frac{u}{\rho},\frac{1}{\rho},0\right)\cdot \begin{pmatrix}
		1\\u\\\frac{1}{2}u^2
	\end{pmatrix}\\
	=&0,
\end{align}
所以第二个特征场是非线性退化的。

\begin{align}
	&\nabla_{\boldsymbol{U}} \lambda_{3}(\boldsymbol{U}) \cdot \boldsymbol{R}^{(3)}(\boldsymbol{U}) \\
	=& \left(-\frac{u}{\rho}+\frac{\gamma}{2a} \frac{(\gamma-1)u^2\rho/2-p}{\rho^2},\frac{1}{\rho}-\frac{\gamma}{2a}\frac{(\gamma-1)u}{\rho},\frac{\gamma}{2a}\frac{(\gamma-1)}{\rho}\right)\cdot \begin{pmatrix}
		1\\u+a\\H+ua
	\end{pmatrix}\\
	=&\frac{\gamma-1}{\rho^2}p+\frac{a}{\rho}>0,
\end{align}
所以第三个特征场是真正非线性的。

\subsection{1b}

一维原始变量形式的Euler方程组为
\begin{equation}
	\begin{cases}
		\rho_{t}+u \rho_{x}+\rho u_{x}=0,\\
		u_{t}+u u_{x}+\frac{1}{\rho} p_{x}=0,\\
		p_{t}+\rho a^{2} u_{x}+u p_{x}=0.	
	\end{cases}
\end{equation}
它属于非守恒形式, 又可以写成矩阵向量形式
\begin{equation}
	\boldsymbol{W}_{t}+\widetilde{\boldsymbol{A}}(\boldsymbol{W}) \boldsymbol{W}_{x}=0, \quad \boldsymbol{W}=\left(\begin{array}{l}
	\rho \\
	u \\
	p
	\end{array}\right), \quad \widetilde{\boldsymbol{A}}=\left(\begin{array}{ccc}
	u & \rho & 0 \\
	0 & u & \frac{1}{\rho} \\
	0 & \rho a^{2} & u
	\end{array}\right)
\end{equation}
其中$\bm{W}$和$\bm{U}$的关系为
\begin{equation}
	\bm{U}_x = \frac{\partial \bm{U}}{\partial \bm{W}} \bm{W}_x,\quad  \bm{U}_t = \frac{\partial \bm{U}}{\partial \bm{W}} \bm{W}_t.
\end{equation}
所以\cref{eq:11}变为
\begin{equation}
	\frac{\partial \bm{U}}{\partial \bm{W}} \bm{W}_x + \boldsymbol{A}(\boldsymbol{U}) \frac{\partial \bm{U}}{\partial \bm{W}} \bm{W}_t =0.
\end{equation}
即
\begin{equation}
	\bm{W}_x + \left(\frac{\partial \boldsymbol{U}}{\partial \boldsymbol{W}}\right)^{-1} \boldsymbol{A}(\boldsymbol{U}) \frac{\partial \bm{U}}{\partial \bm{W}} \bm{W}_t =0.
\end{equation}
所以矩阵 $\widetilde{\boldsymbol{A}}(\boldsymbol{W})$ 与 $\boldsymbol{A}(\boldsymbol{U})$ 相似, $\widetilde{\boldsymbol{A}}(\boldsymbol{W})=\left(\frac{\partial \boldsymbol{U}}{\partial \boldsymbol{W}}\right)^{-1} \boldsymbol{A}(\boldsymbol{U}) \frac{\partial \boldsymbol{U}}{\partial \boldsymbol{W}}.$
由相似性,两矩阵的特征值相等,特征向量有一个乘矩阵的变化,即
\begin{equation}
	\bm{R} = \left(\frac{\partial \boldsymbol{U}}{\partial \boldsymbol{W}}\right) \widetilde{\bm{R}},
\end{equation}
矩阵 $\widetilde{A}$ 的特征值和(左右)特征向量矩阵分别为
\begin{equation}
	\lambda_{1}= u-a, \quad \lambda_{2}=u, \quad \lambda_{3}=u+a,
\end{equation}
\begin{equation}
	\widetilde{\boldsymbol{L}}=	
	\begin{pmatrix}
		0 & 1 & -\frac{1}{\rho a} \\
	1 & 0 & -\frac{1}{a^{2}} \\
	0 & 1 & \frac{1}{\rho a}
	\end{pmatrix},\quad
	\widetilde{\boldsymbol{R}}=
	\begin{pmatrix}
		-\frac{\rho}{2 a} & 1 & -\frac{\rho}{2 a} \\
		\frac{1}{2} & 0 & \frac{1}{2} \\
		-\frac{1}{2} \rho a & 0 & \frac{1}{2} \rho a
	\end{pmatrix}.
\end{equation}
同时,三个特征场的真正非线性和非线性退化的性质保持不变。

\begin{itemize}
	\item 第一个特征场是真正非线性的。
	\item 第二个特征场是非线性退化的。
	\item 第三个特征场是真正非线性的。
\end{itemize}

\subsection{1c}


\begin{equation}
	\bm{U} = \begin{pmatrix}
		\rho \\
	\rho u \\
	E
	\end{pmatrix},\quad
	\begin{array}{c}
	\boldsymbol{R}=\left(\boldsymbol{R}^{(1)}, \boldsymbol{R}^{(2)}, \boldsymbol{R}^{(3)}\right)=\left(\begin{array}{ccc}
	1 & 1 & 1 \\
	u-a & u & u+a \\
	H-u a & \frac{1}{2} u^{2} & H+u a
	\end{array}\right).
	\end{array}
\end{equation}

\begin{equation}
	\frac{\dif u_{1}}{r_{1 j}(\boldsymbol{U})}=\frac{\dif u_{2}}{r_{2 j}(\boldsymbol{U})}=\frac{\dif u_{3}}{r_{3 j}(\boldsymbol{U})}
	\label{eq:12}
\end{equation}

当$j=1$时,
\begin{equation}
	\frac{\dif \rho}{1}=\frac{\dif \rho u}{u-a}=\frac{\dif E}{H-ua}
\end{equation}
化简可得
\begin{equation}
	\begin{cases}
		\rho \dif u+ a\dif \rho=0,\\
		\gamma p\dif \rho=\rho \dif p.
	\end{cases}
\end{equation}
解得
\begin{equation}
	\begin{cases}
		u=-\frac{2\sqrt{\gamma c_1}}{\gamma-1}\rho^{\frac{\gamma-1}{2}}+c_2,\\
		p=c_1\rho^\gamma.
	\end{cases}
\end{equation}

当$j=2$时,
\begin{equation}
	\frac{\dif \rho}{1}=\frac{\dif \rho u}{u}=\frac{\dif E}{\frac{1}{2}u^2}
\end{equation}
化简可得
\begin{equation}
	\begin{cases}
		\rho \dif u=0,\\
		- (\gamma-1) \rho u \dif u= \dif p.
	\end{cases}
\end{equation}
解得
\begin{equation}
	\begin{cases}
		u=c_2,\\
		p=c_1.
	\end{cases}
\end{equation}

当$j=3$时,
\begin{equation}
	\frac{\dif \rho}{1}=\frac{\dif \rho u}{u+a}=\frac{\dif E}{H+ua}
\end{equation}
化简可得
\begin{equation}
	\begin{cases}
		\rho \dif u+ a\dif \rho=0,\\
		\gamma p\dif \rho=\rho \dif p.
	\end{cases}
\end{equation}
解得
\begin{equation}
	\begin{cases}
		u=\frac{2\sqrt{\gamma c_1}}{\gamma-1}\rho^{\frac{\gamma-1}{2}}+c_2,\\
		p=c_1\rho^\gamma.
	\end{cases}
\end{equation}
其中$c_1,c_2$为常数。


\begin{equation}
	r_{1 j}(\boldsymbol{U}) \frac{\partial W}{\partial u_{1}}+r_{2 j}(\boldsymbol{U}) \frac{\partial W}{\partial u_{2}}+r_{3 j}(\boldsymbol{U}) \frac{\partial W}{\partial u_{3}}=0,
	\label{eq:13}
\end{equation}
其中$W=W\left(u_{1}, u_{2}, u_{3}\right) \in \mathbb{R}$.

当$j=1$时,
\begin{equation}
	\frac{\partial W}{\partial u_{1}}+(u-a)\frac{\partial W}{\partial u_{2}}+(H-ua)\frac{\partial W}{\partial u_{3}}=0,
\end{equation}
当$j=2$时,
\begin{equation}
	 \frac{\partial W}{\partial u_{1}}+u \frac{\partial W}{\partial u_{2}}+\frac{1}{2}u^2 \frac{\partial W}{\partial u_{3}}=0,
\end{equation}
当$j=3$时,
\begin{equation}
	 \frac{\partial W}{\partial u_{1}}+(u+a) \frac{\partial W}{\partial u_{2}}+(H+ua) \frac{\partial W}{\partial u_{3}}=0,
\end{equation}

如果连续可微函数$W(u_1,u_2,··· ,u_m)$不恒等于常数, 且在$U$空间中沿 着\cref{eq:12}的任一积分曲线(即特征线), $W$恒为常数, 则称$W$为\cref{eq:12}的一个第一积分.方程组\cref{eq:12}的任一个第一积分是\cref{eq:13}的解.

方程\cref{eq:12}解得出的不变量及其线性组合即为\cref{eq:13}的解。

\section{2}

考虑二维Euler方程组
\begin{equation}
	\frac{\partial}{\partial t} \boldsymbol{U}+\frac{\partial}{\partial x} \boldsymbol{F}(\boldsymbol{U})+\frac{\partial}{\partial y} \boldsymbol{G}(\boldsymbol{U})=0
\end{equation}
其中
\begin{equation}
	\boldsymbol{U}=\left(\begin{array}{c}
		\rho \\
		\rho u \\
		\rho v \\
		E
		\end{array}\right), \quad \boldsymbol{F}(\boldsymbol{U})=\left(\begin{array}{c}
		\rho u \\
		\rho u^{2}+p \\
		\rho u v \\
		u(E+p)
		\end{array}\right), \quad \boldsymbol{G}(\boldsymbol{U})=\left(\begin{array}{c}
		\rho v \\
		\rho u v \\
		\rho v^{2}+p \\
		v(E+p)
		\end{array}\right),
\end{equation}
$p=(\gamma-1) \rho e,\ E=\rho e+\frac{1}{2} \rho\left(u^{2}+v^{2}\right)$.声速 $a=\sqrt{\gamma p / \rho} .$

\subsection{2a}

\begin{equation}
	\frac{\partial \bm{F}}{\partial \bm{U}}=\boldsymbol{A}(\boldsymbol{U})=
	\begin{pmatrix}
	0 & 1 & 0 &0 \\
	-\frac{3y^2}{2x^2} + \gamma \frac{y^2}{2x^2} & \frac{3y}{x} - \frac{\gamma y}{x} & -\frac{(\gamma-1)m}{x}&\gamma-1 \\
	-\frac{ym}{x^2}&\frac{m}{x}&\frac{y}{x}&0\\
	(\gamma-1)\frac{(y^3+ym^2)}{x^3} -\frac{y}{x^2}\gamma z & \frac{\gamma z}{x} - \frac{(\gamma -1)(3y^2+m^2)}{2x^2} &-\frac{y(\gamma-1)m}{x^2}& \gamma \frac{y}{x}
	\end{pmatrix},
\end{equation}

假设
\begin{equation}
	x=\rho,\quad y=\rho u, \quad z=E,\quad m=v.
	\label{eq:20}
\end{equation}
再代入\cref{eq:20}得
\begin{equation}
	\boldsymbol{A}(\boldsymbol{U})=\begin{pmatrix}
		0 & 1 & 0 &0\\
		\frac{\gamma-3}{2} u^{2} + \frac{\gamma-1}{2}v^2 & (3-\gamma) u & - {(\gamma-1)v}&\gamma-1 \\
		-uv&v&u&0\\
		\frac{\gamma-2}{2}u (u^{2}+v^2)-\frac{a^{2} u}{\gamma-1} & \frac{1}{2} (u^{2}+v^2)-(\gamma-1)u^2+\frac{a^{2}}{\gamma-1} &-(\gamma-1)uv &\gamma u
	\end{pmatrix}
\end{equation}
求得特征值为
\begin{equation}
	\lambda_1 = u-a,\quad\lambda_2 = u,\quad\lambda_3 = u,\quad\lambda_4 = u+a.
\end{equation}
特征向量矩阵为
\begin{equation}
	\begin{pmatrix}
		 \frac{2\,{\left(\gamma-1\right)}}{\sigma_2 }&-\frac{2\,v}{\sigma_3 } & \frac{2}{\sigma_3 } & \frac{2\,{\left(\gamma-1\right)}}{\sigma_1 } \\
		-\frac{2\,{\left(a-u\right)}\,{\left(\gamma-1\right)}}{\sigma_2 }& -\frac{2\,u\,v}{\sigma_3 } & \frac{2\,u}{\sigma_3 } & \frac{2\,{\left(a+u\right)}\,{\left(\gamma-1\right)}}{\sigma_1 } \\
		\frac{\sigma_4 }{\sigma_2 } &1 & 0 & \frac{\sigma_4 }{\sigma_1 } \\
		1&0 & 1 & 1 
	\end{pmatrix},
	\label{eq:21}
\end{equation}
其中
\begin{align}
	\sigma_1 &=\gamma u^2 -2au+\gamma v^2 +2a^2 -u^2 -v^2 +2a\gamma u,\\
	\sigma_2 &=2au+\gamma u^2 +\gamma v^2 +2a^2 -u^2 -v^2 -2a\gamma u,\\
	\sigma_3 &=u^2 -v^2, \\
	\sigma_4 &=2v{\left(\gamma-1\right)}.
\end{align}
经验证可得$\lambda_2,\lambda_3$对应的特征场是非线性退化的。$\lambda_1,\lambda_4$对应的特征场是真正非线性的。

\subsection{2b}

可以。由对称性,交换$u,v$符号,并交换$\bm{U},\ \bm{F}$的二三元素即可,所以交换$\bm{A}(\bm{U})$的二三列和二三行(交换顺序并不影响结果)并交换$u,\ v$即得$\bm{B}(\bm{U})$,交换二三列和二三行相当于做了一个相似变换,特征值不变,而交换$u,\ v$后,其特征值为
\begin{equation}
	\lambda_1 = v-a,\quad\lambda_2 = v,\quad\lambda_3 = v,\quad\lambda_4 = v+a.
\end{equation}
交换特征向量矩阵的\cref{eq:21}的二三行并交换$u,\ v$即得$\bm{B}(\bm{U})$的特征向量矩阵。

同样的, $\lambda_2,\lambda_3$对应的特征场是非线性退化的。$\lambda_1,\lambda_4$对应的特征场是真正非线性的。



% \nocite{*}

\input{bib.tex}

\end{document}
