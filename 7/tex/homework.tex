\documentclass[12pt]{article}
\input{/Users/circle/Documents/博一下/homework/setting.tex}
\setcounter{secnumdepth}{2}
\usepackage{autobreak}
\usepackage{amsmath}
\setlength{\parindent}{2em}
\graphicspath{{../}}
\ziju{0.1pt}

%pdf文件设置
\hypersetup{
	pdfauthor={袁磊祺},
	pdftitle={计算流体力学作业7}
}

\title{
		\vspace{-1in} 	
		\usefont{OT1}{bch}{b}{n}
		\normalfont \normalsize \textsc{\LARGE Peking University}\\[0.2cm] % Name of your university/college \\ [25pt]
		\horrule{0.5pt} \\[0.2cm]
		\huge \bfseries{计算流体力学作业7} \\[-0.2cm]
		\horrule{2pt} \\[0.2cm]
}
\author{
		\normalfont 								\normalsize
		College of Engineering \quad 2001111690  \quad 袁磊祺\\	\normalsize
        \today
}
\date{}

\begin{document}

\input{setc.tex}

\maketitle

\section{1}

可以。

例如松弛系统的一阶半离散迎风近似
\begin{gather}
	\frac{\partial}{\partial t} \bm{u}_{j}+\frac{1}{2 h_{j}}\left(\bm{v}_{j+1}-\bm{v}_{j-1}\right)-\frac{1}{2 h_{j}} \bm{A}^{1 / 2}\left(\bm{u}_{j+1}-2 \bm{u}_{j}+\bm{u}_{j-1}\right)=0, \\
	\frac{\partial}{\partial t} \bm{v}_{j}+\frac{1}{2 h_{j}} \bm{A}\left(\bm{u}_{j+1}-\bm{u}_{j-1}\right)-\frac{1}{2 h_{j}} \bm{A}^{1 / 2}\left(\bm{v}_{j+1}-2 \bm{v}_{j}+\bm{v}_{j-1}\right)=-\frac{1}{\varepsilon}\left(\bm{v}_{j}-F\left(\bm{u}_{j}\right)\right).
\end{gather}


考虑一维空间变量的守恒系统
\begin{equation}
	\frac{\partial}{\partial t} \bm{u}+\frac{\partial}{\partial x} \bm{F}(\bm{u})=0, \quad(x, t) \in \mathbb{R}^{1} \times \mathbb{R}, \quad \bm{u} \in \mathbb{R}^{n},
	\label{eq:11}
\end{equation}
其中 $\bm{F}(\bm{u}) \in \mathbb{R}^{n}$ 是一个光滑的向量值函数, 我们引入一个关于\cref{eq:11}的松弛系统
\begin{gather}
\frac{\partial}{\partial t} \bm{u}+\frac{\partial}{\partial x} \bm{v}=0, \quad \bm{v} \in \mathbb{R}^{n}, \\
\frac{\partial}{\partial t} \bm{v}+\bm{A} \frac{\partial}{\partial x} \bm{u}=-\frac{1}{\varepsilon}(\bm{v}-F(\bm{u})), \quad \varepsilon>0.
\end{gather}
其中
\begin{equation}
	\bm{A}=\operatorname{diag}\left\{a_{1}, a_{2}, \ldots, a_{n}\right\},
\end{equation}
是一个需要确定的对角矩阵。对于小的$\varepsilon$, 应用ChapmanEnskog展开得下面的近似
\begin{equation}
	\frac{\partial}{\partial t} \bm{u}+\frac{\partial}{\partial x} \bm{F}(\bm{u})=\varepsilon \frac{\partial}{\partial x}\left(\left(\bm{A}-\bm{F}^{\prime}(\bm{u})^{2}\right) \frac{\partial}{\partial x} \bm{u}\right)
	\label{eq:12}
\end{equation}
其中 $\bm{F}^{\prime}(\bm{u})$ 是 $\bm{F}$的雅可比矩阵.\cref{eq:12} 控制松弛系统的一阶行为。 为了确保\cref{eq:12}的耗散性,则需要
\begin{equation}
	\bm{A}-\bm{F}^{\prime}(\bm{u})^{2} \geqslant 0 \quad \text {对所有的} \quad \bm{u}.
\end{equation}

我们假设$\bm{A}$具有
\begin{equation}
	\bm{A} = a \bm{I},\quad a>0,
\end{equation}
的形式,其中$\bm{I}$是单位矩阵。如果
\begin{equation}
	\frac{\lambda \mu}{a} \leqslant 1 .
\end{equation}
则系统是耗散的,即$\bm{A}$的一个充分条件。


\section{2}

对 Hamilton-Jacobi 方程
\begin{equation}
	\phi_x + H(\phi_x) = 0,
\end{equation}
的半离散格式可以写成
\begin{equation}
	\frac{\dif }{\dif t} \phi_i = - \hat{H}\left(\frac{\Delta^+ \phi_i}{h},\frac{\Delta^- \phi_i}{h}\right),
\end{equation}
其中$\hat{H}$称为数值哈密顿量,它是所有参数的Lipschitz连续函数,并且与PDE中的哈密顿量$H$一致.

相容条件
\begin{equation}
	\hat{H}\left(u,u\right) = H(u).
\end{equation}


\section{3}

Vinokur已经证明:微分类算法(有限差分算法等)和及分类算法(有限体积算法等)仅仅在集合处理上有差别,这种差别会影响到算法到计算精度和效率,但在算法本质上没有根本的差别。所以这里给出二阶精度的Lax-Wendroff格式。\cite[P229]{zhang}

\subsection{Lax-Wendroff}


\begin{align}
\bm{U}\left(x_{j}, y_k, t_{n+1}\right) =&\bm{U}\left(x_{j}, y_k, t_{n}\right)+\tau\left(\bm{U}_{t}\right)_{j,k}^{n}+\frac{1}{2} \tau^{2}\left(\bm{U}_{t t}\right)_{j,k}^{n}+\mathcal{O}\left(\tau^{3}\right) \\
=&\bm{U}\left(x_{j}, y_k, t_{n}\right)-\tau\left(\bm{F}_{x}+\bm{G}_y\right)_{j,k}^{n}\\
& +\frac{1}{2} \tau^{2} \left[\partial_{x}\left(\bm{A}\left(\bm{F}_{x}+\bm{G}_y\right) \right)+\partial_{y}\left(\bm{B}\left(\bm{F}_{x}+\bm{G}_y\right) \right)\right]_{j,k}^{n}+\mathcal{O}\left(\tau^{3}\right) \\
=&\bm{U}\left(x_{j}, y_k, t_{n}\right)-\tau\left(\bm{F}_{x}+\bm{G}_y\right)_{j,k}^{n}\\
&+\frac{1}{2} \tau^{2} \left[\partial_{x}\left(\bm{A}\left(\bm{A}\bm{U}_{x}+\bm{B}\bm{U}_{y}\right) \right)+\partial_{y}\left(\bm{B}\left(\bm{A}\bm{U}_{x}+\bm{B}\bm{U}_{y}\right) \right)\right]_{j,k}^{n}+\mathcal{O}\left(\tau^{3}\right).
\end{align}

利用中心差商代替空间微商, 略去高阶项, 并用 $\bm{U}_{j,k}^{n}$ 代替 $\bm{U}\left(x_{j}, y_k, t_{n}\right)$, 得

\begin{align}
\bm{U}_{j,k}^{n+1}=& \bm{U}_{j,k}^{n}-\frac{\lambda_x}{2}\left(\bm{F}\left(\bm{U}_{j+1,k}^{n}\right)-\bm{F}\left(\bm{U}_{j-1,k}^{n}\right)\right)-\frac{\lambda_y}{2}\left(\bm{G}\left(\bm{U}_{j,k+1}^{n}\right)-\bm{G}\left(\bm{U}_{j,k-1}^{n}\right)\right) \\
&+\frac{\lambda_x^{2} \bm{A}^{2}\left(\bm{U}_{j+\frac{1}{2},k}^{n}\right)}{2}\left(\bm{U}_{j+1,k}^{n}-\bm{U}_{j,k}^{n}\right)-\frac{\lambda_x^{2} \bm{A}^{2}\left(\bm{U}_{j-\frac{1}{2},k}^{n}\right)}{2}\left(\bm{U}_{j,k}^{n}-\bm{U}_{j-1,k}^{n}\right)\\
&+\frac{\lambda_x\lambda_y \bm{A}\bm{B}\left(\bm{U}_{j,k}^{n}\right)}{8}\left(\bm{U}_{j+1,k+1}^{n}-\bm{U}_{j-1,k+1}^{n}\right)-\frac{\lambda_x\lambda_y \bm{A}\bm{B}\left(\bm{U}_{j,k}^{n}\right)}{8}\left(\bm{U}_{j+1,k-1}^{n}-\bm{U}_{j-1,k-1}^{n}\right)\\
&+\frac{\lambda_x\lambda_y \bm{B}\bm{A}\left(\bm{U}_{j,k}^{n}\right)}{8}\left(\bm{U}_{j+1,k+1}^{n}-\bm{U}_{j+1,k-1}^{n}\right)-\frac{\lambda_x\lambda_y \bm{B}\bm{A}\left(\bm{U}_{j,k}^{n}\right)}{8}\left(\bm{U}_{j-1,k+1}^{n}-\bm{U}_{j-1,k-1}^{n}\right)\\
&+\frac{\lambda_y^{2} \bm{B}^{2}\left(\bm{U}_{j,k+\frac{1}{2}}^{n}\right)}{2}\left(\bm{U}_{j,k+1}^{n}-\bm{U}_{j,k}^{n}\right)-\frac{\lambda_y^{2} \bm{B}^{2}\left(\bm{U}_{j,k-\frac{1}{2}}^{n}\right)}{2}\left(\bm{U}_{j,k}^{n}-\bm{U}_{j,k-1}^{n}\right).
\end{align}

其中 $\bm{U}_{j+\frac{1}{2},k}^{n}=\frac{1}{2}\left(\bm{U}_{j,k}^{n}+\bm{U}_{j+1,k}^{n}\right)$.

说明: 其中的 $\bm{A}\left(\bm{U}_{j+\frac{1}{2},k}\right)$ 可以替代为
\begin{equation}
	\bm{A}_{j+\frac{1}{2},k}=\left\{
		\begin{array}{ll}
		\frac{\bm{F}\left(\bm{U}_{j+1,k}\right)-\bm{F}\left(\bm{U}_{j,k}\right)}{\bm{U}_{j+1,k}-\bm{U}_{j,k}}, & \bm{U}_{j+1,k} \neq \bm{U}_{j,k}, \\
		\bm{A}\left(\bm{U}_{j,k}\right), & \bm{U}_{j+1,k}=\bm{U}_{j,k}.
		\end{array}\right.
\end{equation}
$\bm{B}$同理。


\section{4}

\subsection{一维情况}

% 高维的情况中,类似的MUSCL离散方法可以应用在每一维上。考虑多维松弛系统
% \begin{gather}
% \frac{\partial}{\partial t} \bm{u}+\sum_{i=1}^{m} \frac{\partial}{\partial x_{i}} \bm{v}_{i}=0, \\
% \frac{\partial}{\partial t} \bm{v}_{i}+A_{i} \frac{\partial}{\partial x_{i}} \bm{u}=-\frac{1}{\varepsilon}\left(\bm{v}_{i}-F_{i}(\bm{u})\right), \quad i=1,2, \ldots, m.
% \end{gather}
% 对于二维问题,考虑
% \begin{equation}
% 	\begin{aligned}
% 		\frac{\partial}{\partial t} \bm{u}+\frac{\partial}{\partial x} \bm{v}+\frac{\partial}{\partial y} \bm{w} &=0, \\
% 		\frac{\partial}{\partial t} \bm{v}+\bm{A} \frac{\partial}{\partial x} \bm{u} &=-\frac{1}{{\varepsilon}}(\bm{v}-F(\bm{u})),\\
% 		\frac{\partial}{\partial t} \bm{w}+\bm{B} \frac{\partial}{\partial y} \bm{u}&=-\frac{1}{{\varepsilon}}(\bm{w}-G(\bm{u})).
% 		\end{aligned}
% 		\label{eq:41}
% \end{equation}
% Here $A=\operatorname{diag}\left\{a_{1}, \ldots, a_{n}\right\}$ and $B=\operatorname{diag}\left\{b_{1}, \ldots, b_{n}\right)$ with $a_{i}>0, b_{i}>0$ for
% $1 \leqq i \leqq n$
% Suppose the spatial grid points are located at $\left(x_{i+1 / 2}, y_{j+1 / 2}\right)$, the grid widths are $h_{i}^{x}=x_{i+1 / 2}-x_{i-1 / 2}$ and $h_{j}^{y}=y_{j+1 / 2}-y_{j-1 / 2}$ respectively. For a vector $\bm{U}(x, y)$ the point value is $\bm{U}_{i+1 / 2, j+1 / 2}=\bm{U}\left(x_{i+1 / 2}, y_{j+1 / 2}\right)$, the cell average value is defined as
% \begin{equation}
% 	\bm{U}_{i j}=\frac{1}{h_{i}^{x} h_{j}^{y}} \int_{x_{i-1 / 2}}^{x_{i+1 / 2}} \int_{y_{j-1 / 2}}^{y_{j+1 / 2}} \bm{U}(x, y) d x d y .
% \end{equation}

% A conservative, semi-discrete differencing to \cref{eq:41} is
% \begin{gather}
% \frac{\partial}{\partial t} \bm{u}_{i j}+\frac{1}{h_{i}^{x}}\left(\bm{v}_{i+1 / 2, j}-\bm{v}_{i-1 / 2, j}\right)+\frac{1}{h_{j}^{y}}\left(\bm{w}_{i, j+1 / 2}-\bm{w}_{i, j-1 / 2}\right)=0,\\
% \frac{\partial}{\partial t} \bm{v}_{i j}+A \frac{1}{h_{i}^{x}}\left(\bm{u}_{i+1 / 2, j}-\bm{u}_{i-1 / 2, j}\right)=-\frac{1}{\varepsilon}\left(\bm{v}_{i j}-F\left(\bm{u}_{i j}\right)\right), \\
% \frac{\partial}{\partial t} \bm{w}_{i j}+B \frac{1}{h_{j}^{y}}\left(\bm{u}_{i, j+1 / 2}-\bm{u}_{i, j-1 / 2}\right)=-\frac{1}{\varepsilon}\left(\bm{w}_{i j}-G\left(\bm{u}_{i j}\right)\right).
% \end{gather}
% Applying the MUSCL discretization to $\bm{v} \pm A^{1 / 2} \bm{u}$ in the $x$-direction and $\bm{w} \pm B^{1 / 2} \bm{u}$ in the ${y}$-direction respectively, for the $p$-th component,
% \begin{align}
% u_{i+1 / 2, j}=& \frac{1}{2}\left(u_{i j}+u_{i+1, j}\right)-\frac{1}{2 \sqrt{a_{p}}}\left(v_{i+1, j}-v_{i j}\right) \\
% &+\frac{1}{4 \sqrt{a_{p}}}\left(h_{i}^{x} \sigma_{i j}^{x,+}+h_{i+1}^{x} \sigma_{i+1, j}^{x_{i}-},\right),\\
% v_{i+1 / 2, j}=& \frac{1}{2}\left(v_{i j}+v_{i+1, j}\right)-\frac{\sqrt{a_{p}}}{2}\left(u_{i+1, j}-u_{i j}\right) \\
% &+\frac{1}{4}\left(h_{i}^{x} \sigma_{i j}^{x_{i}+}-h_{i+1}^{x}{\sigma_{i+1, j}}^{x_{i j}}\right), \\
% u_{i, j+1 / 2}=& \frac{1}{2}\left(u_{i j}+u_{i, j+1}\right)-\frac{1}{2 \sqrt{b_{p}}}\left(w_{i, j+1}-w_{i j}\right) \\
% &+\frac{1}{4 \sqrt{b_{p}}}\left(h_{j}^{y} \sigma_{i j}^{y_{i j}}+h_{j+1}^{y} \sigma_{i, j+1}^{y,-}\right), \\
% w_{i, j+1 / 2}=& \frac{1}{2}\left(w_{i j}+w_{i, j+1}\right)-\frac{\sqrt{b_{p}}}{2}\left(u_{i, j+1}-u_{i j}\right) \\
% &+\frac{1}{4}\left(h_{j}^{y} \sigma_{i j}^{y_{i j}+}-h_{j+1}^{y}{_{i, j+1}}^{y,-}\right).
% \end{align}
% The slope limiters are defined as
% \begin{gather}
% \sigma_{i j}^{x, \pm}=\frac{1}{h_{i}^{x}}\left(v_{i+1, j} \pm \sqrt{a_{p}} u_{i+1, j}-v_{i j} \mp \sqrt{a_{p}} u_{i j}\right) \phi\left(\theta_{i j}^{x, \pm}\right), \\
% \theta_{i j}^{x, \pm}=\frac{v_{i j} \pm \sqrt{a_{p}} u_{i j}-v_{i-1, j} \mp \sqrt{a_{p}} u_{i-1, j}}{v_{i+1, j} \pm \sqrt{a_{p}} u_{i+1, j}-v_{i j} \mp \sqrt{a_{p}} u_{i j}}, \\
% \sigma_{i j}^{y, \pm}=\frac{1}{h_{j}^{y}}\left(w_{i, j+1} \pm \sqrt{b_{p}} u_{i, j+1}-w_{i j} \mp \sqrt{b_{p}} u_{i j}\right) \phi\left(\theta_{i j}^{y, \pm}\right), \\
% \theta_{i j}^{y, \pm}=\frac{w_{i j} \pm \sqrt{b_{p}} u_{i j}-w_{i, j-1} \mp \sqrt{b_{p}} u_{i, j-1}}{w_{i, j+1} \pm \sqrt{b_{p}} u_{i, j+1}-w_{i j} \mp \sqrt{b_{p}} u_{i j}}.
% \end{gather}


这里考虑标量的情况。
\begin{equation}
	\begin{aligned}
		\frac{\partial}{\partial t} {u}+\frac{\partial}{\partial x} {v}&=0, \quad {v} \in \mathbb{R}^{1}, \\
		\frac{\partial}{\partial t} {v}+{a} \frac{\partial}{\partial x} {u}&=-\frac{1}{\varepsilon}({v}-F({u})), \quad \varepsilon>0.
	\end{aligned}
	\label{eq:42}
\end{equation}

松弛系统\cref{eq:42}的均匀网格的简单一阶守恒格式可以写成
\begin{equation}
	\begin{aligned}
		\frac{u_{j}^{n+1}-u_{j}^{n}}{\tau}+ \frac{v_{j+1 / 2}^{n}-v_{j-1 / 2}^{n}}{h}&=0, \\
		\frac{v_{j}^{n+1}-v_{j}^{n}}{\tau}+a \frac{u_{j+1 / 2}^{n}-u_{j-1 / 2}^{n}}{h} &=-\frac{1}{\varepsilon}\left(v_{j}^{n+1}-f\left(u_{j}^{n+1}\right)\right).
		\end{aligned}
		\label{eq:412}
\end{equation}


\cref{eq:42}的一阶迎风格式是
\begin{equation}
	\begin{aligned}
		u_{j+1 / 2}^{n}=&\frac{1}{2}\left(u_{j}^{n}+u_{j+1}^{n}\right)-\frac{1}{2 \sqrt{a}}\left(v_{j+1}^{n}-v_{j}^{n}\right), \\
		v_{j+1 / 2}^{n}=&\frac{1}{2}\left(v_{j}^{n}+v_{j+1}^{n}\right)+\frac{\sqrt{a}}{2}\left(u_{j+1}^{n}-u_{j}^{n}\right).
	\end{aligned}
\end{equation}
应用\cref{eq:412}得
\begin{equation}
	\frac{u_{j}^{n+1}-u_{j}^{n}}{k}+\frac{1}{2 h}\left(v_{j+1}^{n}-v_{j-1}^{n}\right)-\frac{\sqrt{a}}{2 h}\left(u_{j+1}^{n}-2 u_{j}^{n}+u_{j-1}^{n}\right)=0.
\end{equation}
\begin{equation}
	\frac{v_{j}^{n+1}-v_{j}^{n}}{k} +\frac{a}{2 h}\left(u_{j+1}^{n}-u_{j-1}^{n}\right)-\frac{1}{2 \sqrt{a} h}\left(v_{j+1}^{n}-2 v_{j}^{n}+v_{j-1}^{n}\right)=-\frac{1}{\varepsilon}\left(v_{j}^{n}-f\left(u_{j}^{n}\right)\right).
\end{equation}


Using a Hilbert expansion gives the leading order equations (as $\left.\varepsilon \rightarrow 0^{+}\right)$,
\begin{equation}
	\begin{aligned}
		v_{j}^{n} &=f\left(u_{j}^{n}\right), \\
		u_{j}^{n+1} &=u_{j}^{n}-\frac{\lambda}{2}\left(f\left(u_{j+1}^{n}\right)-f\left(u_{j-1}^{n}\right)\right)+\frac{\sqrt{a} \lambda}{2}\left(u_{j+1}^{n}-2 u_{j}^{n}+u_{j-1}^{n}\right).
	\end{aligned}
	\label{eq:43}
\end{equation}
where
\begin{equation}
	\lambda=\frac{k}{h}.
\end{equation}

The scheme in (\ref{eq:43}) is a first-order relaxed scheme which is a Lax-Friedrichs type scheme. A numerical scheme
\begin{equation}
	U_{j}^{n+1}=\mathbf{H}\left(U^{n} ; j\right)
\end{equation}
is called a monotone scheme if
\begin{equation}
	\frac{\partial}{\partial U_{i}^{n}} \mathbf{H}\left(U^{n} ; j\right) \geqq 0 \quad \text { for all } \quad i, j, U^{n}.
\end{equation}
It can be checked easily that (\ref{eq:43}) is a monotone scheme provided the standard CFL condition arising from the discrete convection terms
\begin{equation}
	\sqrt{a} \lambda \leqq 1.
	\label{eq:44}
\end{equation}
where $\lambda = \frac{\tau}{h}$ and the subcharacteristic condition 
\begin{equation}
	-\sqrt{a} \leq f^{\prime}(u) \leq \sqrt{a} \quad \text { for all } u
	\label{eq:45}
\end{equation}
are satisfied. Thus we have the following theorem.

Under the CFL condition (\ref{eq:44}) and the subcharacteristic condition (\ref{eq:45}), the relaxed scheme (\ref{eq:43}) is a monotone scheme.

所以时间步长应该满足\cref{eq:44}。

\subsection{二维情况}

对于二维问题,考虑标量情况
\begin{equation}
	\begin{aligned}
		\frac{\partial}{\partial t} {u}+\frac{\partial}{\partial x} {v}+\frac{\partial}{\partial y} {w} &=0, \\
		\frac{\partial}{\partial t} {v}+{a} \frac{\partial}{\partial x} {u} &=-\frac{1}{{\varepsilon}}({v}-F({u})),\\
		\frac{\partial}{\partial t} {w}+{b} \frac{\partial}{\partial y} {u}&=-\frac{1}{{\varepsilon}}({w}-G({u})).
		\end{aligned}
		\label{eq:46}
\end{equation}
松弛系统\cref{eq:46}的均匀网格的简单一阶守恒格式可以写成
\begin{equation}
	\begin{aligned}
		\frac{u_{j,k}^{n+1}-u_{j,k}^{n}}{\tau}+ \frac{v_{j+\frac{1}{2},k}^{n}-v_{j-\frac{1}{2},k}^{n}}{h_x}+ \frac{w_{j,k+\frac{1}{2}}^{n}-w_{j,k-\frac{1}{2}}^{n}}{h_y}&=0, \\
		\frac{v_{j,k}^{n+1}-v_{j,k}^{n}}{\tau}+a \frac{u_{j+\frac{1}{2},k}^{n}-u_{j-\frac{1}{2},k}^{n}}{h_x} &=-\frac{1}{\varepsilon}\left(v_{j,k}^{n+1}-f\left(u_{j,k}^{n+1}\right)\right),\\
		\frac{w_{j,k}^{n+1}-w_{j,k}^{n}}{\tau}+b \frac{u_{j,k+\frac{1}{2}}^{n}-u_{j,k-\frac{1}{2}}^{n}}{h_x} &=-\frac{1}{\varepsilon}\left(w_{j,k}^{n+1}-f\left(u_{j,k}^{n+1}\right)\right).
		\end{aligned}
	\label{eq:47}
\end{equation}
一阶迎风格式是
\begin{equation}
	\begin{aligned}
		u_{j+1 / 2}^{n}=&\frac{1}{2}\left(u_{j}^{n}+u_{j+1}^{n}\right)-\frac{1}{2 \sqrt{a}}\left(v_{j+1}^{n}-v_{j}^{n}\right), \\
		v_{j+1 / 2}^{n}=&\frac{1}{2}\left(v_{j}^{n}+v_{j+1}^{n}\right)+\frac{\sqrt{a}}{2}\left(u_{j+1}^{n}-u_{j}^{n}\right).
	\end{aligned}
\end{equation}
特征条件
\begin{gather}
	-\sqrt{a} \leq F^{\prime}(u) \leq \sqrt{a} \quad \text { for all } u,\\
	-\sqrt{b} \leq G^{\prime}(u) \leq \sqrt{b} \quad \text { for all } u.
\end{gather}

时间步长需要满足
\begin{gather}
	\sqrt{a} \lambda_x \leqq 1,\\
	\sqrt{b} \lambda_y \leqq 1.
\end{gather}
这样格式是单调的。


% \nocite{*}

\input{bib.tex}

\end{document}
