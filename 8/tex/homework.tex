\documentclass[12pt]{article}
\input{/Users/circle/Documents/博一下/homework/setting.tex}
\setcounter{secnumdepth}{2}
\usepackage{autobreak}
\usepackage{amsmath}
\setlength{\parindent}{2em}
\graphicspath{{../}}
\ziju{0.1pt}

%pdf文件设置
\hypersetup{
	pdfauthor={袁磊祺},
	pdftitle={计算流体力学作业8}
}

\title{
		\vspace{-1in} 	
		\usefont{OT1}{bch}{b}{n}
		\normalfont \normalsize \textsc{\LARGE Peking University}\\[0.2cm] % Name of your university/college \\ [25pt]
		\horrule{0.5pt} \\[0.2cm]
		\huge \bfseries{计算流体力学作业8} \\[-0.2cm]
		\horrule{2pt} \\[0.2cm]
}
\author{
		\normalfont 								\normalsize
		College of Engineering \quad 2001111690  \quad 袁磊祺\\	\normalsize
        \today
}
\date{}

\begin{document}

\input{/Users/circle/Documents/博一下/homework/setc.tex}

\maketitle

\section{A}

\subsection{1}

考虑1D双曲守恒律方程
\begin{equation}
	u_{t}+f(u)_{x}=0,\ x \in \mathbb{R},\ t>0,
\end{equation}
$u(x, 0)=u_{0}(x)$, 和均匀网格 $\left\{x_{j}: x_{j}=j h,\ j \in \mathbb{Z}\right\} .$ 试写出上述方程的
守恒形式的Engquist-Osher格式. 它是否是单调格式? 如果是, 是否可以证明它满足极值原理?如果应用这些格式于具体问题的数值计算,
则时间步长如何选取?


迎风格式的特征是: 根据信息传播的方向离散微分方程. 考虑标量方程
\begin{equation}
	u_{t}+f_{x}=0,
	\label{eq:11}
\end{equation}
如果 $f^{\prime}(u)>0,\ \forall u \in \mathbb{R}$, 则
\begin{equation}
	\left.\frac{\dif u}{\dif t}\right|_{j}+\frac{1}{h}\left(f_{j}-f_{j-1}\right)=0,
\end{equation}
或如果 $f^{\prime}(u)<0,\ \forall u \in \mathbb{R}$, 则
\begin{equation}
	\left.\frac{\dif u}{\dif t}\right|_{j}+\frac{1}{h}\left(f_{j+1}-f_{j}\right)=0,
\end{equation}
一般地, \cref{eq:11}可改写为
\begin{equation}
	u_{t}+f_{x}^{+}+f_{x}^{-}=0,
\end{equation}
其中 $f^{+}(u)+f^{-}(u)=f(u),\ \frac{\dif f^{+}}{\dif u} \geq 0,\ \frac{\dif f^{-}}{\dif u} \leq 0$, 因而可离散为
\begin{equation}
	\left.\frac{\dif u}{\dif t}\right|_{j}+\frac{1}{h}\left(f_{j}^{+}-f_{j-1}^{+}\right)+\frac{1}{h}\left(f_{j+1}^{-}-f_{j}^{-}\right)=0,
\end{equation}
进一步离散为
\begin{align}
	u_{j}^{n+1} & = \ u_{j}^{n}-\frac{\Delta t}{\Delta x}\left(\Delta_{+} f^{-}\left(u_{j}^{n}\right)+\Delta_{-} f^{+}\left(u_{j}^{n}\right)\right) \\
	            & =\ H(u^n_{j-1},u^n_{j},u^n_{j+1}),
	\label{eq:13}
\end{align}
其中
\begin{equation}
	f^{+}(u)=\int_{0}^{u} \max \left(0, f^{\prime}(\xi)\right) \dif \xi+f(0), \quad f^{-}(u)=\int_{0}^{u} \min \left(0, f^{\prime}(\xi)\right) \dif \xi+f(0).
\end{equation}
考虑格式的单调性,
\begin{equation}
	\frac{\partial u^{n+1}_j}{\partial u^{n}_j} = 1 + r\min \left(0, f^{\prime}(u^{n}_j)\right) - r\max \left(0, f^{\prime}(u^{n}_j)\right) = 1 - r\abs{f^{\prime}(u^{n}_j)}.
\end{equation}
其中
\begin{equation}
	r = \frac{\Delta t}{\Delta x}.
\end{equation}

易得另外两个偏导
\begin{equation}
	\frac{\partial u^{n+1}_j}{\partial u^{n}_{j+1}} = - r\min \left(0, f^{\prime}(u^{n}_{j+1})\right) \geq 0,\quad \frac{\partial u^{n+1}_j}{\partial u^{n}_{j-1}} =  r\max \left(0, f^{\prime}(u^{n}_{j-1})\right) \geq 0.
\end{equation}
当满足CFL
\begin{equation}
	1 \geq r\abs{f^{\prime}(u)}
	\label{eq:12cfl}
\end{equation}
条件时,格式是单调的。

对于局部极值原理,根据\cref{eq:13}有
\begin{equation}
	H(u,u,u) = u,
	\label{eq:14}
\end{equation}
又根据之前得到的格式的单调性有
\begin{align}
	H(u_{j-1},u_{j},u_{j+1}) \leq H(u,u,u) = u,\quad u = \max{\left\{u_{j-1},u_{j},u_{j+1}\right\}}, \\
	H(u_{j-1},u_{j},u_{j+1}) \geq H(u,u,u) = u,\quad u = \min{\left\{u_{j-1},u_{j},u_{j+1}\right\}}.
\end{align}
所以格式在\cref{eq:12cfl}条件下是满足局部极值原理的。

时间步长需要满足\cref{eq:12cfl}。对每一步来说
\begin{equation}
	\tau \leq \frac{h}{\max_j\left\{\abs{f'(u^n_j)}\right\}}.
\end{equation}


\subsection{2}


上述格式可以看成是对双曲守恒律方程先空间离散为
\begin{equation}
	\frac{\dif u_{j}}{\dif t}=-\frac{1}{h}\left(\hat{f}_{j+\frac{1}{2}}-\hat{f}_{j-\frac{1}{2}}\right)=: L\left(u_{j}(t)\right),
\end{equation}
[属于线方法或半离散方法],然后再对上式中的时间导数采用显 式Euler方法离散. 如果把时间离散换成下列Runge-Kutta方法
\begin{align}
	u^{(1)} & =u^{n}+\Delta t L\left(u^{n}\right)                                               \\
	u^{n+1} & =\frac{1}{2} u^{n}+\frac{1}{2} u^{(1)}+\frac{1}{2} \Delta t L\left(u^{(1)}\right)
\end{align}
或
\begin{align}
	u^{(1)} & =u^{n}+\Delta t_{n} L\left(u^{n}\right)                                               \\
	u^{(2)} & =\frac{3}{4} u^{n}+\frac{1}{4}\left(u^{(1)}+\Delta t_{n} L\left(u^{(1)}\right)\right) \\
	u^{n+1} & =\frac{1}{3} u^{n}+\frac{2}{3}\left(u^{(2)}+\Delta t_{n} L\left(u^{(2)}\right)\right)
\end{align}
前述结果又如何?


定义
\begin{equation}
	H\left(u^n,j\right) = H \left(u_{j-1},u_{j},u_{j+1}\right).
\end{equation}

\begin{equation}
	u^{(1)}_j  =u^{n}_j +\Delta t_{n} L\left(u^{n}_j \right) = H(u^n,j),
\end{equation}

\begin{equation}
	\begin{aligned}
		u^{n+1} & =\frac{1}{2} u^{n}+\frac{1}{2} u^{(1)}+\frac{1}{2} \Delta t L\left(u^{(1)}\right)                                                                                                                        \\
		        & =\frac{1}{2} u^{n}+\frac{1}{2}H\left(u^{(1)}_j\right)                                                                                                                                                    \\
		        & =  \frac{1}{2} u^{n}+ \frac{1}{2} H\left(H\left(u_{j-2}^{n}, u_{j-1}^{n}, u_{j}^{n}\right), H\left(u_{j-1}^{n}, u_{j}^{n}, u_{j+1}^{n}\right), H\left(u_{j}^{n}, u_{j+1}^{n}, u_{j+2}^{n}\right)\right),
	\end{aligned}
\end{equation}
定义
\begin{equation}
	H\left(H\left(u_{j-2}^{n}, u_{j-1}^{n}, u_{j}^{n}\right), H\left(u_{j-1}^{n}, u_{j}^{n}, u_{j+1}^{n}\right), H\left(u_{j}^{n}, u_{j+1}^{n}, u_{j+2}^{n}\right)\right) =: H^{(1)}\left(u^n,j\right).
\end{equation}
在\cref{eq:12cfl} 条件下$H$是单调的,所以复合函数也是单调的,故格式是单调的。

根据\cref{eq:14}易知
\begin{equation}
	H\left(H(u,u,u),H(u,u,u),H(u,u,u)\right) = H(u,u,u) = u.
\end{equation}
同理可得格式是满足局部极值原理的。

而对于
\begin{align}
	u^{(1)} & =u^{n}+\Delta t_{n} L\left(u^{n}\right)                                                \\
	u^{(2)} & =\frac{3}{4} u^{n}+\frac{1}{4}\left(u^{(1)}+\Delta t_{n} L\left(u^{(1)}\right)\right)  \\
	u^{n+1} & =\frac{1}{3} u^{n}+\frac{2}{3}\left(u^{(2)}+\Delta t_{n} L\left(u^{(2)}\right)\right),
\end{align}

\begin{equation}
	u^{(1)} =u^{n}+\Delta t_{n} L\left(u^{n}\right)=H\left(u^n,j\right).
\end{equation}
\begin{equation}
	u^{(2)} =\frac{3}{4} u^{n}+\frac{1}{4}\left(u^{(1)}+\Delta t_{n} L\left(u^{(1)}\right)\right)=\frac{3}{4} u^{n}+\frac{1}{4}H\left(u^{(1)},j\right).
\end{equation}
\begin{equation}
	u^{n+1} =\frac{1}{3} u^{n}+\frac{2}{3}\left(u^{(2)}+\Delta t_{n} L\left(u^{(2)}\right)\right) = \frac{1}{3} u^{n}+\frac{2}{3}H\left(u^{(2)},j\right).
\end{equation}
由此可见,在\cref{eq:12cfl}的条件下该格式同样有单调性,且有\cref{eq:14}的性质,所以满足局部极值原理。


\section{B}

考虑2D双曲守恒律方程
\begin{equation}
	u_{t}+f(u)_{x}+g(u)_{y}=0, x, y \in \mathbb{R}, t>0,
\end{equation}

$f=g=\frac{1}{2} u^{2},\ u(x, y, 0)=u_{0}(x, y)$, 和均匀的矩形网格$\{\left(x_{j}, y_{k}\right):x_{j}=j h_{x},\ y_{k}=k h_{y},\ j,\ k \in \mathbb{Z}\}.$ 请详细写出上述方程的如下形式的Godunov格式
\begin{equation}
	\begin{aligned}
		\bar{u}_{j, k}^{n+1} =\  & \bar{u}_{j, k}^{n}-\frac{\tau}{h_{x}}\left[f\left(\omega\left(0 ; \bar{u}_{j, k}^{n}, \bar{u}_{j+1, k}^{n}\right)\right)-f\left(\omega\left(0 ; \bar{u}_{j-1, k}^{n}, \bar{u}_{j, k}^{n}\right)\right)\right] \\
		                         & -\frac{\tau}{h_{y}}\left[g\left(\omega\left(0 ; \bar{u}_{j, k}^{n} ,\bar{u}_{j, k+1}^{n}\right)\right)-g\left(\omega\left(0 ; \bar{u}_{j, k-1}^{n}, \bar{u}_{j, k}^{n}\right)\right)\right],
	\end{aligned}
\end{equation}
其中 $\omega \left(\frac{x-x_{j+\frac{1}{2}}}{t-t_n};\bar{u}_{j, k}^{n}, \bar{u}_{j+1, k}^{n}\right)$是
\begin{equation}
	u_{t}+f(u)_{x}=0,\ u\left(x, t_{n}\right)=\left\{\begin{array}{ll}
		\bar{u}_{j, k}^{n},   & x-x_{j+\frac{1}{2}}<0, \\
		\bar{u}_{j+1, k}^{n}, & x-x_{j+\frac{1}{2}}>0,
	\end{array}\right.
\end{equation}
的精确解; $\omega\left(\frac{y-y_{k+\frac{1}{2}}}{t-t_{n}} ; \bar{u}_{j, k}^{n}, \bar{u}_{j, k+1}^{n}\right)$ 是
\begin{equation}
	u_{t}+g(u)_{y}=0,\ u\left(y, t_{n}\right)=\left\{\begin{array}{ll}
		\bar{u}_{j, k}^{n},   & y-y_{k+\frac{1}{2}}<0, \\
		\bar{u}_{j, k+1}^{n}, & y-y_{k+\frac{1}{2}}>0,
	\end{array}\right.
\end{equation}
的精确解。也就是给出局部1D Riemann问题精确解, 再完整地给
出 $f(\omega(0 ; \bar{u}_{j, k}^{n}, \bar{u}_{j+1, k}^{n}))$ 和 $g(\omega(0 ; \bar{u}_{j, k}^{n}, \bar{u}_{j, k+1}^{n}))$ 的计算. [提示: 参照课堂上写的1D Burgers方程的Godunov格式]


对于 1D Burgers 方程的黎曼问题
\begin{equation}
	u_{t}+uu_x=0,\ u\left(x, 0\right)=\left\{\begin{array}{ll}
		u_l, & x<0, \\
		u_r, & x>0,
	\end{array}\right.
\end{equation}
的精确解为
\begin{enumerate}
	\item $u_l>u_r$,为激波解
	      \begin{equation}
		      u\left(x, 0\right)=\left\{\begin{array}{ll}
			      u_l, & x<ct, \\
			      u_r, & x>ct,
		      \end{array}\right.
	      \end{equation}
	      其中$c=\frac{u_l+u_r}{2}$.
	\item $u_l<u_r$,为稀疏波解
	      \begin{equation}
		      u\left(x, 0\right)=\left\{\begin{array}{ll}
			      u_l, & x<u_lt,      \\
			      x/t, & u_lt<x<u_rt, \\
			      u_r, & x>u_rt.
		      \end{array}\right.
	      \end{equation}
\end{enumerate}

下面分类讨论来求$x=0$的精确解, $\omega\left(0;u_l,u_r\right)=:\varphi(u_l,u_r).$

\begin{enumerate}
	\item $u_l>u_r$,
	      \begin{enumerate}
		      \item $c>0,\ \omega\left(0;u_l,u_r\right)=u_l.$
		      \item $c<0,\ \omega\left(0;u_l,u_r\right)=u_r.$
	      \end{enumerate}
	\item $u_l<u_r$,
	      \begin{enumerate}
		      \item $u_l>0,\ \omega\left(0;u_l,u_r\right)=u_l.$
		      \item $u_r<0,\ \omega\left(0;u_l,u_r\right)=u_r.$
		      \item $u_l<0<u_r,\ \omega\left(0;u_l,u_r\right)=0.$
	      \end{enumerate}
\end{enumerate}

\begin{equation}
	f(\omega(0 ; \bar{u}_{j, k}^{n}, \bar{u}_{j+1, k}^{n}))=f(\varphi(\bar{u}_{j, k}^{n}, \bar{u}_{j+1, k}^{n}))=\frac{1}{2}\varphi^2(\bar{u}_{j, k}^{n}, \bar{u}_{j+1, k}^{n}),
\end{equation}
\begin{equation}
	g(\omega(0 ; \bar{u}_{j, k}^{n}, \bar{u}_{j, k+1}^{n}))=g(\varphi(\bar{u}_{j, k}^{n}, \bar{u}_{j, k+1}^{n}))=\frac{1}{2}\varphi^2(\bar{u}_{j, k}^{n}, \bar{u}_{j, k+1}^{n}).
\end{equation}

\begin{equation}
	\begin{aligned}
		\bar{u}_{j, k}^{n+1} =\  & \bar{u}_{j, k}^{n}-\frac{\tau}{h_{x}}\left(\frac{1}{2}\varphi^2\left( \bar{u}_{j, k}^{n}, \bar{u}_{j+1, k}^{n}\right)-\frac{1}{2}\varphi^2\left( \bar{u}_{j-1, k}^{n}, \bar{u}_{j, k}^{n}\right)\right) \\
		                         & -\frac{\tau}{h_{y}}\left(\frac{1}{2}\varphi^2\left( \bar{u}_{j, k}^{n} ,\bar{u}_{j, k+1}^{n}\right)-\frac{1}{2}\varphi^2\left( \bar{u}_{j, k-1}^{n}, \bar{u}_{j, k}^{n}\right)\right).
	\end{aligned}
\end{equation}



\section{C}

证明1D完全气体Euler方程组的1激波的关系式, 即讲义 [CFDLect06-com03_cn.pdf] 的第27页的1激波的关系式.


\begin{proof}

	以$s_1$速度行进的1激波(假设朝左)关系式,令
	\begin{equation}
		\hat{u}_{L}=u_{L}-s_{1}, \quad \hat{u}_{*}=u_{*}-s_{1}, \quad M_{L}=u_{L} / a_{L}, \quad M_{S}=s_{1} / a_{1}.
	\end{equation}

	在这个新框架下,应用RH条件, 有
	\begin{equation}
		\begin{aligned}
			\rho_{*} \hat{u}_{*}                      & =\rho_{L} \hat{u}_{L},                      \\
			\rho_{*} \hat{u}_{*}^{2}+p_{*}            & =\rho_{L} \hat{u}_{L}^{2}+p_{L},            \\
			\hat{u}_{*}\left(\hat{E}_{*}+p_{*}\right) & =\hat{u}_{L}\left(\hat{E}_{L}+p_{L}\right),
		\end{aligned}
		\label{eq:31}
	\end{equation}
	其中 $\hat{E}_{*}=\rho_{*} e_{*}+\frac{1}{2} \rho_{*} \hat{u}_{*}^{2} .$
	\cref{eq:31}中第三式的左端和右端分别可改写为
	\begin{equation}
		\hat{u}_{*} \rho_{*}\left[\frac{1}{2} \hat{u}_{*}^{2}+\left(e_{*}+\frac{p_{*}}{\rho_{*}}\right)\right], \quad \hat{u}_{L} \rho_{L}\left[\frac{1}{2} \hat{u}_{L}^{2}+\left(e_{L}+\frac{p_{L}}{\rho_{L}}\right)\right].
	\end{equation}
	应用\cref{eq:31}中的第一式和比含 $h=\frac{p}{\rho}+e$, 则有
	\begin{equation}
		\frac{1}{2} \hat{u}_{*}^{2}+h_{*}=\frac{1}{2} \hat{u}_{L}^{2}+h_{L},
		\label{eq:311}
	\end{equation}
	又由\cref{eq:31}中的第一式和第二式, 得
	\begin{equation}
		\begin{aligned}
			\rho_{*} \hat{u}_{*}^{2}=\left(\rho_{L} \hat{u}_{L}\right) \hat{u}_{L}+p_{L}-p_{*} \stackrel{\cref{eq:31}}{\implies } \rho_{*} \hat{u}_{*} \frac{\rho_{*} \hat{u}_{*}}{\rho_{L}}+p_{L}-p_{*}, \\
			\Rightarrow \rho_{*}^{2} \hat{u}_{*}^{2}\left(\frac{\rho_{L}-\rho_{*}}{\rho_{*} \rho_{L}}\right)=p_{L}-p_{*}, \Rightarrow \hat{u}_{*}^{2}=\left(\frac{\rho_{L}}{\rho_{*}}\right)\left(\frac{p_{L}-p_{*}}{\rho_{L}-\rho_{*}}\right) .
		\end{aligned}
		\label{eq:32}
	\end{equation}
	类似地,有
	\begin{equation}
		\hat{u}_{L}^{2}=\left(\frac{\rho_{*}}{\rho_{L}}\right)\left[\frac{p_{L}-p_{*}}{\rho_{L}-\rho_{*}}\right],
		\label{eq:33}
	\end{equation}
	将\cref{eq:32,eq:33} 代入\cref{eq:311}, 得
	\begin{equation}
		h_{*}-h_{L}=\frac{1}{2}\left(p_{*}-p_{L}\right)\left[\frac{\rho_{*}+\rho_{L}}{\rho_{*} \rho_{L}}\right],
	\end{equation}
	或
	\begin{equation}
		e_{*}-e_{L}=\frac{1}{2}\left(p_{*}+p_{L}\right)\left[\frac{\rho_{*}-\rho_{L}}{\rho_{*} \rho_{L}}\right], \quad h=e+\frac{p}{\rho}.
		\label{eq:34}
	\end{equation}

	注意: 直到此, 没有用到状态方程 $e=e(p, \rho)$ 的具体形式. 下面将仅考虑完全气 体 $e=\frac{p}{\rho(\gamma-1)} .$ 应用其及 \cref{eq:34}, 可得
	\begin{equation}
		\frac{\rho_{*}}{\rho_{L}}=\frac{\left(\frac{p_{*}}{p_{L}}\right)+\left(\frac{\gamma-1}{\gamma+1}\right)}{\left(\frac{\gamma-1}{\gamma+1}\right)\left(\frac{p_{*}}{p_{L}}\right)+1},
		\label{eq:35}
	\end{equation}
	这建立了穿过激波的密度比 $\left(\frac{\rho_{*}}{\rho_{L}}\right)$ 和压力比 $\left(\frac{p_{*}}{p_{L}}\right)$ 之间的一个有用关系式.


	引入Mach数
	\begin{equation}
		M_{L}=\frac{u_{L}}{a_{L}}, \quad \text{激波前流动的马赫数, 在老的框架里,}
	\end{equation}
	\begin{equation}
		M_{S}=\frac{s_{3}}{a_{L}}, \quad \text{激波马赫数.}
		\label{eq:36}
	\end{equation}
	由\cref{eq:33,eq:34,eq:35}, 可以给出穿过激波的密度比和压力比关于相对马赫 数 $M_{L}$ 和激波马赫数 $M_{S}$ 的差的函数表示式(激波关系):
	\begin{align}
		\frac{\rho_{*}}{\rho_{L}} & =\frac{(\gamma+1)\left(M_{L}-M_{S}\right)^{2}}{(\gamma-1)\left(M_{L}-M_{S}\right)^{2}+2}, \\
		\frac{p_{*}}{p_{L}}       & =\frac{2 \gamma\left(M_{L}-M_{S}\right)^{2}-(\gamma-1)}{(\gamma+1)}.
		\label{eq:37}
	\end{align}
	又由\cref{eq:37}, 得下列关系式(根号前的符号由Lax激波不等式决定)
	\begin{equation}
		M_{L}-M_{S}=\sqrt{\left(\frac{\gamma+1}{2 \gamma}\right)\left(\frac{p_{*}}{p}\right)+\left(\frac{\gamma-1}{2 \gamma}\right)},
	\end{equation}
	或
	\begin{equation}
		s_{3}=u_{L}-a_{L} \sqrt{\left(\frac{\gamma+1}{2 \gamma}\right)\left(\frac{p_{*}}{p}\right)+\left(\frac{\gamma-1}{2 \gamma}\right)} .
	\end{equation}

	由\cref{eq:31}中的第一式, 有
	\begin{equation}
		u_{*}=\left(1-\frac{\rho_{L}}{\rho_{*}}\right) s_{3}+\left(\frac{\rho_{L}}{\rho_{*}}\right) u_{L}.
	\end{equation}

	所以有
	\begin{align}
		\frac{\rho_{*}}{\rho_{L}}         & =\frac{\left(\frac{p_{*}}{p_{L}}\right)+\left(\frac{\gamma-1}{\gamma+1}\right)}{\left(\frac{\gamma-1}{\gamma+1}\right)\left(\frac{p_{*}}{p_{L}}\right)+1}, \\
		\frac{\rho_{*}}{\rho_{L}}         & =\frac{(\gamma+1)\left(M_{L}-M_{S}\right)^{2}}{(\gamma-1)\left(M_{L}-M_{S}\right)^{2}+2},                                                                  \\
		\frac{p_{*}}{p_{L}}               & =\frac{2 \gamma\left(M_{L}-M_{S}\right)^{2}-(\gamma-1)}{(\gamma+1)},                                                                                       \\
		u_{*}                             & =\left(1-\frac{\rho_{L}}{\rho_{*}}\right) s_{1}+\left(\frac{\rho_{L}}{\rho_{*}}\right) u_{L},                                                              \\
		s_{1}                             & =u_{L}-a_{L} \sqrt{\frac{\gamma+1}{2 \gamma}\left(\frac{p_{*}}{p_{L}}\right)+\left(\frac{\gamma-1}{2 \gamma}\right)},                                      \\
		M_{L}                      -M_{S} & =+\sqrt{\left(\frac{\gamma+1}{2 \gamma}\right)\left(\frac{p_{*}}{p_{L}}\right)+\left(\frac{\gamma-1}{2 \gamma}\right)}.
	\end{align}


\end{proof}


% \nocite{*}

% \input{bib.tex}

\end{document}


% Engquist-Osher格式(Math. Comput., 34(1980), 45-75; $36(1981)$, 321-352)
% \begin{equation}
% 	\hat{\boldsymbol{F}}\left(\boldsymbol{U}_{j}, \boldsymbol{U}_{j+1}\right)=\frac{1}{2}\left(\boldsymbol{F}\left(\boldsymbol{U}_{j}\right)+\boldsymbol{F}\left(\boldsymbol{U}_{j+1}\right)\right)-\frac{1}{2} \int_{\boldsymbol{U}_{j}}^{\boldsymbol{U}_{j+1}}|\boldsymbol{A}(\boldsymbol{U})| \dif \boldsymbol{U}
% \end{equation}
% 其中, 积分路径为 $\Gamma_{j}=\cup_{i=1}^{m} \Gamma_{j}^{(i)}$, 曲线 $\Gamma_{j}^{(i)}$ 为与 $A$ 的右特征向量 $\boldsymbol{R}^{(i)}$ 平行的线段, 即
% \begin{equation}
% 	\Gamma_{j}^{(i)}:\left\{\begin{array}{l}
% 		\frac{\dif \boldsymbol{U}^{(i)}}{\dif s}=\boldsymbol{R}^{(i)}\left(\boldsymbol{U}^{(i)}\right), \quad 0 \leq s \leq s_{j}^{(i)} \text { 或 }\  0 \geq s \geq s_{j}^{(i)}, \\
% 		\boldsymbol{U}^{(i)}(s=0)=\boldsymbol{U}^{(i+1)}\left(s_{j}^{(i+1)}\right),
% 	\end{array}\right.
% \end{equation}
% 其中 $i=m, \cdots, 1,\ \boldsymbol{U}^{(m+1)}\left(s_{j}^{(m+1)}\right)=\boldsymbol{U}_{j},\ \boldsymbol{U}^{(1)}\left(s_{j}^{(1)}\right)=\boldsymbol{U}_{j+1}$.