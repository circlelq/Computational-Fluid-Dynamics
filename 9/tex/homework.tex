\documentclass[12pt]{article}
\input{/Users/circle/Documents/博一下/homework/setting.tex}
\setcounter{secnumdepth}{2}
\usepackage{autobreak}
\usepackage{amsmath}
\setlength{\parindent}{2em}
\graphicspath{{../}}
\ziju{0.1pt}

%pdf文件设置
\hypersetup{
	pdfauthor={袁磊祺},
	pdftitle={计算流体力学作业9}
}

\title{
		\vspace{-1in} 	
		\usefont{OT1}{bch}{b}{n}
		\normalfont \normalsize \textsc{\LARGE Peking University}\\[0.2cm] % Name of your university/college \\ [25pt]
		\horrule{0.5pt} \\[0.2cm]
		\huge \bfseries{计算流体力学作业9} \\[-0.2cm]
		\horrule{2pt} \\[0.2cm]
}
\author{
		\normalfont 								\normalsize
		College of Engineering \quad 2001111690  \quad 袁磊祺\\	\normalsize
        \today
}
\date{}

\begin{document}

%%%%%%%%%%%%%%%%%%%%%%%%%%%%%%%%%%%%%%%%%%%%%%
\captionsetup[figure]{name={图},labelsep=period}
\captionsetup[table]{name={表},labelsep=period}
\renewcommand\contentsname{目录}
\renewcommand\listfigurename{插图目录}
\renewcommand\listtablename{表格目录}
\renewcommand\refname{参考文献}
\renewcommand\indexname{索引}
\renewcommand\figurename{图}
\renewcommand\tablename{表}
\renewcommand\abstractname{摘\quad 要}
\renewcommand\partname{部分}
\renewcommand\appendixname{附录}
\def\equationautorefname{式}%
\def\footnoteautorefname{脚注}%
\def\itemautorefname{项}%
\def\figureautorefname{图}%
\def\tableautorefname{表}%
\def\partautorefname{篇}%
\def\appendixautorefname{附录}%
\def\chapterautorefname{章}%
\def\sectionautorefname{节}%
\def\subsectionautorefname{小小节}%
\def\subsubsectionautorefname{subsubsection}%
\def\paragraphautorefname{段落}%
\def\subparagraphautorefname{子段落}%
\def\FancyVerbLineautorefname{行}%
\def\theoremautorefname{定理}%
\crefname{figure}{图}{图}
\crefname{equation}{式}{式}
\crefname{table}{表}{表}
%%%%%%%%%%%%%%%%%%%%%%%%%%%%%%%%%%%%%%%%%%%

\maketitle

\section{1}

取$(x_j,x_{j+1})\times [t_n,t_{n+1})$,对$u_t+f_x=0$积分,在$\lambda \max \limits_{u} \abs{f'(u)} \leq \frac{1}{2}$时,能得到什么格式?

可以得到Godunov格式。

定义 $I_{j}:=\left(x_{j-\frac{1}{2}}, x_{j+\frac{1}{2}}\right)$, 设
\begin{equation}
	\frac{\tau}{h} \max _{u}\left\{\left|f^{\prime}(u)\right|\right\} \leq \frac{1}{2}.
	\label{eq:11}
\end{equation}


\begin{enumerate}
	\item 计算初始函数的单元平均值:
	      \begin{equation}
		      u_{j}^{0}=\frac{1}{h} \int_{x_{j-\frac{1}{2}}}^{x_{j+\frac{1}{2}}} u(x, 0) \dif x.
	      \end{equation}
	\item 对于 $n \geq 0$, 由单元平均定义(重构) $t_{n}$ 时刻的近似解(分片常数函数)
	      \begin{equation}
		      u_{h}\left(x, t_{n}\right):=u_{j}^{n}, \quad x \in I_{j},\ t \in\left[t_{n}, t_{n+1}\right),
	      \end{equation}

	      并精确地解(局部)Riemann问题
	      \begin{equation}
		      \left\{\begin{array}{l}
			      \frac{\partial u}{\partial t}+\frac{\partial f}{\partial x}=0, \quad t \in\left[t_{n}, t_{n+1}\right) \\
			      u\left(x, t_{n}\right)=\left\{\begin{array}{ll}
				      u_{h}\left(x_{j+\frac{1}{2}}-0, t_{n}\right), & x<x_{j+\frac{1}{2}} \\
				      u_{h}\left(x_{j+\frac{1}{2}}+0, t_{n}\right), & x>x_{j+\frac{1}{2}}
			      \end{array}\right.
		      \end{array}\right.
		      \label{eq:12}
	      \end{equation}


	      用 $\omega(x, t)$ 表示\cref{eq:12}的精确解. 特别地, $\omega(x, t)$ 具有形式:
	      \begin{equation}
		      \omega(x, t)=\omega\left(\frac{x-x_{j+\frac{1}{2}}}{t-t_{n}}, u_{h}\left(x_{j+\frac{1}{2}}-0, t_{n}\right), u_{h}\left(x_{j+\frac{1}{2}}+0, t_{n}\right)\right),
	      \end{equation}
	\item 计算单元平均:
	      \begin{equation}
		      u_{j}^{n+1}=\frac{1}{h} \int_{x_{j-\frac{1}{2}}}^{x_{j+\frac{1}{2}}} \omega\left(x, t_{n+1}\right) \dif x.
	      \end{equation}

	      在CFL条件\cref{eq:11}下, 有
	      \begin{equation}
		      u_{j}^{n+1}=  \frac{1}{h} \int_{x_{j}-\frac{1}{2} h}^{x_{j}} \omega\left(\frac{x-x_{j-\frac{1}{2}}}{\tau}, u_{j-1}^{n}, u_{j}^{n}\right) \dif x  +\frac{1}{h} \int_{x_{j}}^{x_{j}+\frac{1}{2} h} \omega\left(\frac{x-x_{j+\frac{1}{2}}}{\tau}, u_{j}^{n}, u_{j+1}^{n}\right) \dif x.
	      \end{equation}
\end{enumerate}








\section{2}


\begin{equation}
	u_t + f_x+g_x=0
\end{equation}
在矩形网格$\left\{x_j=jh,\ y_k=kh\right\}$上可离散为
\begin{equation}
	u^{n+1}_{j,k} = u^n_{j,k} - \lambda_x\left(\hat{f}^n_{j+\frac{1}{2},k}-\hat{f}^n_{j-\frac{1}{2},k}\right) - \lambda_y \left(\hat{g}^n_{j,k+\frac{1}{2}}-\hat{g}^n_{j,k-\frac{1}{2}}\right),
	\label{eq:21}
\end{equation}
其中$\hat{f}^n_{j-\frac{1}{2},k}$形式可以是一维方程对应格式的数值通量, $\hat{g}^n_{j,k+\frac{1}{2}}$类似

问题:
\begin{align}
	\hat{f}^n_{j+\frac{1}{2},k} = & \frac{1}{2}\left(f_{j,k} + f_{j+1,k}\right)-\frac{1}{2\lambda_x}\left(u_{j+1,k}-u_{j,k}\right)
	\label{eq:22}                                                                                                                  \\
	\hat{g}^n_{j,k+\frac{1}{2}} = & \frac{1}{2}\left(g_{j,k} + g_{j,k+1}\right)-\frac{1}{2\lambda_x}\left(u_{j,k+1}-u_{j,k}\right)
	\label{eq:23}
\end{align}
是否合适?

将\cref{eq:22,eq:23}代入\cref{eq:21}可得
\begin{equation}
	\begin{aligned}
		u^{n+1}_{j,k} = & \ u^n_{j,k} - \frac{\lambda_x}{2}\left({f}^n_{j+1,k}-{f}^n_{j-1,k}\right) -  \frac{\lambda_y}{2}\left({g}^n_{j,k+1}-{g}^n_{j,k-1}\right) \\
		                & +\frac{1}{2}\left(u^n_{j+1,k}-2u^n_{j,k}+u^n_{j-1,k}\right)+\frac{1}{2}\left(u^n_{j,k+1}-2u^n_{j,k}+u^n_{j,k-1}\right).
	\end{aligned}
	\label{eq:24}
\end{equation}
这相当于在不稳定的中心差分格式上加了一个耗散项。

假设
\begin{equation}
	{f}^n_{j+1,k}-{f}^n_{j-1,k} = f'\left(u^n_{j+1,k}-u^n_{j-1,k}\right),
\end{equation}
\begin{equation}
	{g}^n_{j,k+1}-{g}^n_{j,k-1} = g' \left(u^n_{j,k+1}-u^n_{j,k-1}\right).
\end{equation}

将
\begin{equation}
	u^n_{j,k} = \xi^n_l \me^{\mi \beta (j+k)h}
\end{equation}
代入方程得放大因子为
\begin{equation}
	G(\beta,h) = -1 + 2\cos(\beta h) -\left[\lambda(f'+g')\sin(\beta h)\right]\mi.
\end{equation}
是恒不稳定的。













% \nocite{*}

% \bibliographystyle{plain}

\phantomsection

\addcontentsline{toc}{section}{参考文献} %向目录中添加条目,以章的名义
\bibliography{homework}

\end{document}

