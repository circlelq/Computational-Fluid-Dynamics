\documentclass[12pt]{article}
\input{/Users/circle/Documents/博一下/homework/setting.tex}
\setcounter{secnumdepth}{2}
\usepackage{autobreak}
\usepackage{amsmath}
\setlength{\parindent}{2em}
\graphicspath{{../}}
\ziju{0.1pt}

%pdf文件设置
\hypersetup{
	pdfauthor={袁磊祺},
	pdftitle={计算流体力学作业6}
}

\title{
		\vspace{-1in} 	
		\usefont{OT1}{bch}{b}{n}
		\normalfont \normalsize \textsc{\LARGE Peking University}\\[0.2cm] % Name of your university/college \\ [25pt]
		\horrule{0.5pt} \\[0.2cm]
		\huge \bfseries{计算流体力学作业6} \\[-0.2cm]
		\horrule{2pt} \\[0.2cm]
}
\author{
		\normalfont 								\normalsize
		College of Engineering \quad 2001111690  \quad 袁磊祺\\	\normalsize
        \today
}
\date{}

\begin{document}

\input{setc.tex}

\maketitle

\section{1}

编写一维完全气体Euler方程组的LF格式, MacCormack 格式, 和一阶
精度的显式迎风格式(Roe格式)的程序, 并计算讲义 (CFDLect04-
com01_cn.pdf) 的第101-102页的问题2和问题4.

\begin{equation}
	\left\{\begin{array}{c}
	\left(\begin{array}{c}
	\rho \\
	\rho u \\
	E
	\end{array}\right)_{t}+\left(\begin{array}{c}
	\rho u \\
	\rho u^{2}+p \\
	u(E+p)
	\end{array}\right)_x=0, \\
	p=(\gamma-1)\left(E-\frac{1}{2} \rho u^{2}\right),\ \gamma=1.4.
	\end{array}\right.
\end{equation}

LF 格式
\begin{equation}
	\bm{u}^{n+1}_{j} = \frac{1}{2} \left( \bm{u}^{n}_{j+1} + \bm{u}^{n}_{j-1} \right) - \frac{1}{2} r \left( \bm{f}^{n}_{j+1} - \bm{f}^{n}_{j-1} \right).
\end{equation}
其中$r=\tau/h$,稳定性条件为
\begin{equation}
	\abs{\bm{u}}_{\max} \frac{\tau}{h} \leqslant 1.
\end{equation}










































\input{bib.tex}

\end{document}
