\documentclass[12pt]{article}
\input{/Users/circle/Documents/博一下/homework/setting.tex}
\setcounter{secnumdepth}{2}
\usepackage{autobreak}
\usepackage{amsmath}
\setlength{\parindent}{2em}
\graphicspath{{../}}
\ziju{0.1pt}

%pdf文件设置
\hypersetup{
	pdfauthor={袁磊祺},
	pdftitle={计算流体力学上机作业2}
}

\title{
		\vspace{-1in} 	
		\usefont{OT1}{bch}{b}{n}
		\normalfont \normalsize \textsc{\LARGE Peking University}\\[0.2cm] % Name of your university/college \\ [25pt]
		\horrule{0.5pt} \\[0.2cm]
		\huge \bfseries{计算流体力学上机作业2} \\[-0.2cm]
		\horrule{2pt} \\[0.2cm]
}
\author{
		\normalfont 								\normalsize
		College of Engineering \quad 2001111690  \quad 袁磊祺\\	\normalsize
        \today
}
\date{}

\begin{document}

%%%%%%%%%%%%%%%%%%%%%%%%%%%%%%%%%%%%%%%%%%%%%%
\captionsetup[figure]{name={图},labelsep=period}
\captionsetup[table]{name={表},labelsep=period}
\renewcommand\contentsname{目录}
\renewcommand\listfigurename{插图目录}
\renewcommand\listtablename{表格目录}
\renewcommand\refname{参考文献}
\renewcommand\indexname{索引}
\renewcommand\figurename{图}
\renewcommand\tablename{表}
\renewcommand\abstractname{摘\quad 要}
\renewcommand\partname{部分}
\renewcommand\appendixname{附录}
\def\equationautorefname{式}%
\def\footnoteautorefname{脚注}%
\def\itemautorefname{项}%
\def\figureautorefname{图}%
\def\tableautorefname{表}%
\def\partautorefname{篇}%
\def\appendixautorefname{附录}%
\def\chapterautorefname{章}%
\def\sectionautorefname{节}%
\def\subsectionautorefname{小小节}%
\def\subsubsectionautorefname{subsubsection}%
\def\paragraphautorefname{段落}%
\def\subparagraphautorefname{子段落}%
\def\FancyVerbLineautorefname{行}%
\def\theoremautorefname{定理}%
\crefname{figure}{图}{图}
\crefname{equation}{式}{式}
\crefname{table}{表}{表}
%%%%%%%%%%%%%%%%%%%%%%%%%%%%%%%%%%%%%%%%%%%

\maketitle

编写一维完全气体Euler方程组的LF格式, MacCormack 格式, 和一阶精度的显式迎风格式(Roe格式)的程序, 并计算讲义 (CFDLect04-com01_cn.pdf) 的第101-102页的问题2和问题4.












\bibliographystyle{plain}

\phantomsection

\addcontentsline{toc}{section}{参考文献} %向目录中添加条目,以章的名义
\bibliography{homework}

\end{document}
